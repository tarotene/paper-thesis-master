% !TeX root = Body.tex

\chapter{Velocity-driven Non-equilibrium Phase Transition in Ising Models}\label{ch:review}

To discuss the non-equilibrium crossover between two different dimensionalities, we make a brief review of the exact results~\cite{Hucht2009b} by Hucht. His analysis is based on the fact that two Ising cylinders with relative motion make a novel mean field, which leads the system to a non-trivial phase transition.

Let us consider two equivalent square lattices of the Ising model each of which contacts the other by one of its one-dimensional boundaries (see Fig.~\ref{fig:SketchNEIsing}). We make one lattice slide along the contact plane against the other lattice with a constant velocity $v$. The entire system thereby goes into a non-equilibrium stationary state instead of equilibration. The non-equilibrium stationary state well describes the behavior of two magnetic materials with a friction. This setup is explained in detail in Chapter~\ref{chap:NumSim}.

\begin{figure}[htbp]
	\centering
	\subcaptionbox[0.5\linewidth]{\label{subcap:one-dim}}{\raisebox{17.5mm}{\includegraphics[width=0.4\linewidth]{1dIsingAfterTwoWalks.pdf}}}
	\subcaptionbox[0.5\linewidth]{\label{subcap:two-dim}}{\includegraphics[width=0.4\linewidth]{2dIsingAfterTwoWalks.pdf}}
	\caption{Sketches of the models considered in Ref.~\cite{Hucht2009b}. Both cases depict a schematic view after the sliding by twice a lattice constant. (a) Two chains of the one-dimensional model. (b) Two lattices of the two-dimensional model.}
	\label{fig:SketchNEIsing}
\end{figure}

As a well known fact, the ordinary two-dimensional Ising model has an equilibrium phase transition at the critical temperature $T_{\rm c,eq}=2/(\log\left[1+\sqrt{2}\right])$ in the thermodynamical limit. The system with the friction becomes equivalent to the equilibrium case in the limit of $v\to 0$. In addition to the ordinary phase transition, a novel phase transition in which the magnetization grows on the sliding boundary (see Fig.~\ref{fig:NEPTinIsing}). Now we denote the velocity dependent non-equilibrium critical point by $T_{\rm c}(v)$ apart from the equilibrium critical point $T_{\rm eq, c}$. Hucht~\cite{Hucht2009b} claims that the critical temperature $T_{\rm c}(v)$ \textit{deviates} from $T_{\rm c,eq}$ at the point $v=0$ towards the limit $v=\infty$.

\begin{figure}[htbp]
\centering
\includegraphics[width=0.5\linewidth]{NEPTIsing.pdf}
\caption{The phase diagram of the two-dimensional non-equilibrium Ising model~\cite{Hucht2009b}. The black solid line, the red dashed line and the blue solid line indicate the ordinary bulk phase transition, a non-equilibrium boundary phase transition for the Metropolis rate and the multiplicative rate, respectively. From right to left across the non-equilibrium phase boundary the system acquires non-zero expectation value of magnetization on the sliding boundary.}
\label{fig:NEPTinIsing}
\end{figure}

This phenomenon wes first reported in the numerical results~\cite{Kadau2008} by Kadau \textit{et al}.\ using Monte Carlo simulations both on the Metropolis and the Glauber algorithms in a two-dimensional model, and then was investigated in a more analytic manner~\cite{Hucht2009b} by Hucht in several dimensionalities and model geometries. One of the important points of the latter result is that in the limit $v\to \infty$ we can write down a closed exact equation for the \textit{second} critical temperature $T^{*}_{\rm c}(\infty)$. It is also important that a novel algorithm called \textit{multipricative rate} enabled us to give an equation of $T^{*}_{\rm c}(v)$, which depends on the flip rate and the sliding velocity $v>0$.

If the velocity $v$ is much less than the rate $\xi^{(\rm eq)}_{x}(\beta)/\tau^{(\rm eq)}_{x}(\beta)$, we can expect the system to behave similarly to its equilibrium state, where $\xi^{(\rm eq)}_{x}(\beta)$ and $\tau^{(\rm eq)}_{x}(\beta)$ are the correlation length along the direction parallel to the sliding surface and the correlation time, respectively, for the equilibrium state at an inverse temperature $\beta:=(k_{\rm B}T)^{-1}$. This corresponds to the case in which the pumped energy by the constant sliding quickly relaxes into the heat bath and the structure of domain walls near the sliding boundary is well sustained. On the other hand, the velocity $v$ much greater than the rate $\xi^{(\rm eq)}_{x}(\beta)/\tau^{(\rm eq)}_{x}(\beta)$ should lead the system to a stationary state far from equilibrium, so that the structure near the sliding boundary is destroyed. 

In the latter case a mean field picture well describes the behavior of the system; a set of the moving spins along the contact plane act on the other set of \textit{relatively} moving spins as a spatially averaged effective field. This view enables us to write a self-consistent equation for the temperature $T_{\rm c}(\infty)$.

We summarize the result for one-dimensional chains and two-dimensional planes in order to discuss the crossover from one dimension to two dimensions in our models in Chapter~\ref{chap:Summary}. We first give a general Hamiltonian of the Ising model as follows.
\begin{align}
\beta \mathcal{H}_{\mu}:=-K\sum_{i<j}\sigma_{i}\sigma_{j}-h^{\rm ext}\sum_{i}\sigma_{i}-\sum_{i}k_{i}\mu_{i}\sigma_{i},
\end{align}
where $K$, $h^{\rm ext}$ and $k_{i}\mu_{i}$ denote the exchange interaction, the external field and the stochastic field on the $i$th spin ($\mu_{i}=\pm 1$), respectively. The geometry of the model is either Fig.~\ref{fig:SketchNEIsing} (a) or (b). Now we assume that $\mu_{i}$ obeys a probability distribution $p_{i}(\mu_{i})$ such that $\langle\mu_{i}\rangle:=\sum_{\mu_{i}=\pm 1}p_{i}(\mu_{i})\mu_{i}=m_{i}$ for a given  value of $m_{i}$. The form $p_{i}(\mu_{i}):=(1+\mu_{i}m_{i})/2$ actually satisfies the condition. 

If we decompose the Hamiltonian into the contribution of the stochastic field and the rest as
\begin{align}
\beta \mathcal{H}_{\mu}=&\beta \mathcal{H}_{0} - \sum_{i}k_{i}\mu_{i}\sigma_{i},
\end{align}
where
\begin{align}
\beta \mathcal{H}_{0}:=&-K\sum_{i<j}\sigma_{i}\sigma_{j}-\sum_{i}h^{\rm ext}_{i}\sigma_{i},
\end{align}
the partition function of the system is written in the form
\begin{align}
\mathcal{Z} = \left\langle\mathrm{Tr}_{\sigma}\left[\mathrm{e}^{-\beta \mathcal{H}_{\mu}}\right]\right\rangle =&\mathrm{Tr}_{\sigma}\left[\mathrm{e}^{-\beta \mathcal{H}_{0}}\left\langle\prod_{i}\mathrm{e}^{k_{i}\mu_{i}\sigma_{i}}\right\rangle\right]\\
=&\prod_{j}\cosh k_{j}\;\mathrm{Tr}_{\sigma}\left[\mathrm{e}^{-\beta \mathcal{H}_{0}}\prod_{j}(1+\sigma_{i}m_{i}\tanh k_{i})\right].\label{eq:Z}
\end{align}
Owing to the translation invariance along the sliding direction, we assume a homogeneous boundary magnetization $m_{i}=m_{\rm b}$ for all boundary sites $i$. Each boundary magnetization acts as an effective field $h_{\rm b}$ on the other boundary magnetization. 

The model therefore reduces to
\begin{align}
\beta\mathcal{H}_{\rm eq}:=-K\sum_{i<j}\sigma_{i}\sigma_{j}-h_{\rm b}\sum_{i}\sigma_{i}=\beta\mathcal{H}_{0} - \sum_{i}b\sigma_{i},
\end{align}
with $b:=h_{\rm b}-h^{\rm ext}$, which relaxes toward the equilibrium state. Its partition function is written as
\begin{align}
\mathcal{Z}_{\rm eq}=\mathrm{Tr}_{\sigma}\left[\mathrm{e}^{-\beta\mathcal{H}_{\rm eq}}\right]=&\mathrm{Tr}_{\sigma}\left[\mathrm{e}^{\beta \mathcal{H}_{0}}\sum_{i}\mathrm{e}^{b\sigma_{i}}\right]\\
=&\prod_{i}\cosh b\;\mathrm{Tr}_{\sigma}\left[\mathrm{e}^{-\beta\mathcal{H}_{0}}\prod_{i}(1+\sigma_{i}\tanh b)\right].\label{eq:Zeq}
\end{align}
Comparing the right-hand sides of Eqs.~\eqref{eq:Z} and \eqref{eq:Zeq}, we have $\mathcal{Z} \propto\mathcal{Z}_{\rm eq}$ if it holds that $\tanh b=m_{\rm b}\tanh k_{i}$.

Our model gives the strength of the stochastic field $k_{i}$ as the exchange interaction across the sliding surface $K$, and thus it holds that $k_{i}=K$ and
\begin{align}
h_{\rm b}=\tanh^{-1}\left(m_{\rm b}\tanh K\right),
\end{align}
in the limit $h^{\rm ext}\to 0$.
The boundary magnetization under a static field $h_{\rm b}$ has a form of
\begin{align}
m_{\rm b, eq}(K,h_{\rm b}):=\frac{\partial}{\partial h_{\rm b}}\mathcal{Z}_{\rm eq},
\end{align}
and thus we have
\begin{align}
m_{\rm b, eq}\left(K,\tanh^{-1}\left(m_{b}\tanh K\right)\right)=m_{b}\label{cond:critcond1}
\end{align}
as a self-consistent relation for $m_{b}$. The critical point is given by
\begin{align}
1=\left.\frac{\partial m_{\rm b,eq}}{\partial m_{\rm b}}\right|_{m_{\rm b}=0}.\label{rel:CritTilt}
\end{align}
Expanding the left-hand side of Eq.~\eqref{cond:critcond1} to the first order of $m_{\rm b}$ and using the condition \eqref{rel:CritTilt} we have
\begin{align}
m_{\rm b,eq}\left(K,0\right) + m_{b}\tanh\left(K\right)\left.\frac{\partial m_{\rm b,eq}}{\partial h_{\rm b}}\right|_{h_{\rm b}=0} = m_{b}.
\end{align}
With the definition of the equilibrium susceptibility $\chi_{\rm b, eq}(K):=\left.\partial m_{\rm b,eq}/\partial h_{\rm b}\right|_{h_{\rm b}=0}$ and the fact $m_{\rm b,eq}\left(K,0\right)=0$, we finally have
\begin{align}
\tanh \left(K\right)\chi^{(0)}_{\rm b, eq}(K) = 1.\label{cond:critcond2}
\end{align}
The condition \eqref{cond:critcond2} determines the non-equilibrium critical temperature $K=K_{\rm c}$. Using the expression of the magnetization for the one-dimensional Ising model and the boudnary magnetization for the two-dimensional model, we have the numerical solutions of the condition \eqref{cond:critcond2} as
\begin{align}
{K_{\rm c}}^{-1} = \begin{cases}
2.2691853\dots\quad\text{in one dimension,}\\
2.6614725\dots\quad\text{in two dimensions.}
\end{cases}
\end{align}

Note that the temperature $K_{\rm c}$ corresponds to the non-equilibrium critical temperature in the limit $v\to\infty$, and is consistent to the extrapolation from the results for two dimensions with the multipricative rate (see Fig.~\ref{fig:NEPTinIsing}).