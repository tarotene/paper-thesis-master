% !TeX root = Body.tex

\chapter{Velocity-driven Non-equilibrium Phase Transition in Ising Models}\label{ch:review}

To discuss the non-equilibrium crossover between two different dimensions, we have to make a brief review the exact results\cite{Hucht2009b} by Hucht. Their analysis is based on the fact that two Ising cylinders with relative motion make a novel mean field and they lead the system to non-trivial phase transition.

We now consider two equivalent square lattices of Ising model each of which contacts with the other by one of its one-dimensional boundary. If we leave two models alone, we can regard the models as an Ising model with a twice the size. With the coupling to the heat bath, we expect the model thermalize. But one of the couple are moving along the contact plane with a constant velocity $v$, the entire model goes into a non-equilibrium stationary state, instead of thermalization. The non-equilibrium stationary state well describes the behavior of two magnetic materials are sliding with the friction. This setup are explained in detail in Chapter \ref{chap:NumSim}.

The two dimensional Ising model has the phase transition which occurs at the critical temperature $T_{\rm c}=\text{\textcolor{red}{exact value or equation form}}$ in the thermodynamical limit $N/V\to\infty$ with $N/V=\text{const.}$, where $N$ is the number of spins and $V$ is the volume of the system as a well known fact. The model which we consider in this chapter is no exception, if the sliding velocity $v$ is set to zero. The result\cite{Hucht2009b} claims that the critical temperature $T_{\rm c}$ \textit{branches} at the point $v=0$ towards the limit $v=\infty$. The system exhibits the ordinary phase transition at the vertical line $T=T_{\rm c}(v=0)$, whereas exhibits a novel phase transition at the curve $T=T_{\rm c}(v>0)$ where the magnetization grows up near the contact plane rather than the rest \textcolor{red}{cite the phase diagram of the ref.}. These phenomena were first reported in the numerical results\cite{Kadau2008} by Kadau et al. using Monte Carlo simulations with both Metropolis algorithms and Glauber algorithms in two-dimensional models, and then investigated in more exact manner\cite{Hucht2009b} by Hucht in several dimensions and model geometries. One of the important point in the latter result is that for $v\to 0$ we can write the closed exact equation for the \textit{second} critical temperature $T_{\rm c}(v>0)$. In addition it is also important that a novel algorithm, called \textit{multipricative rate}, enabled us to give the equation for $T_{\rm c}(v>0)$ which fully depends on the algorithm for arbitrary $v>0$.

If the velocity $v$ is much less than the rate $\xi^{(\rm eq)}_{x}(\beta)/\tau^{(\rm eq)}_{x}(\beta)$, we can expect the system to behave well similarly to its equilibrium state. We denote the correlation length along the direction parallel to the contact plane by $\xi^{(\rm eq)}_{x}(\beta)$ and the correlation time by $\tau^{(\rm eq)}_{x}(\beta)$ respectively for the equilibrium state of an inverse temperature $\beta:=(k_{\rm B}T)^{-1}$. This corresponds to the circumstance that the pumped energy by the constant sliding is most quickly relaxed towards the heat bath. In the case, the structure of domain walls near the contact plane is well sustained. On the other hand, the velocity $v$ much greater than the rate $\xi^{(\rm eq)}_{x}(\beta)/\tau^{(\rm eq)}_{x}(\beta)$ will lead the system to a stationary state much far from equilibrium, then the structure near the contact plane will be destroyed. 

In the latter case that the velocity is much higher, a mean field picture that two sets of the moving spins along the contact plane act on the other \textit{relatively} moving spins as the spatially-averaged effective field. In other words, the magnetization of one half of the system corresponds the effective field of the other, and vice versa. This enables us to write a self-consistent equation for the temperature $T_{\rm c}(v\to\infty)$.

We summarize the result for one-dimensional chains and that of two-dimensional planes, in order to discuss the crossover from one-dimension to two-dimension in our models in Chapter \ref{chap:Summary}.

\textcolor{red}{Brief review.}