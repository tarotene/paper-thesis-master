% !TeX root = Body.tex

\chapter{Analysis based on Stochastic Matrices}
\label{chap:ProofEx}

In this chapter, we prove the existence of the non-equilibrium stationary state in our model with a half sliding for arbitrary velocities $v>0$. In the first section, we discuss the formulation by stochastic matrices, and then see Monte Carlo simulations and the stochastic matrices are equivalent in terms of the probability with a simple example. In the following two sections, we prove several facts for stochastic matrices, which ensures that almost all Monte Carlo simulations converge to a equilibrium state and its uniqueness, and then the properties are taken over and lead to a unique stationary state even if the matrix contains a kind of perturbation factors. In the last two sections, we propose the way to construct the matrix for both equilibrium cases and non-equilibrium stationary cases, and discuss the distributions of their eigenvalues in terms of the convergence.

\section{A Simple Example: Stochastic Ising Model with $N$-spins}
Monte Carlo simulations for lattice spin systems extract the relevant subspace from the full space instead of an exact calculation of the partition function using a stochastic process. The subspace depends on given parameters and if we use the canonical distribution with a fixed temperature $T$, the temperature determines the subspace. The stochastic process is expressed as a trajectory of variables by the time in the subspace. Averaging the trajectory for an enough long time, we can compute physical quantities with any desired accuracy.

We now consider a matrix form of the stochastic process. For example, the one-dimensional Ising chain with $N$-spins has $2^{N}$ states. If we labeled each of the states by $i=1,2,\dots,2^{N}$, we can write the stochastic time evolution of the system by a set of the existence probabilities $\{p_{i}(t)\}$ such that the system is in the $i$-th state at a time $t$, and the transition probabilities $T_{ij}$ such that the system in the $j$-th state changes to the $i$-th state. Note that not the transition probabilities $T_{ij}$ but the existence probabilities $\{p_{i}(t)\}$ play the role of time evolution.

We additionally define the conditional probability $\tilde{p}_{ij}(t)$ such that the system in the $j$-th state at a time $t$ changes to the $i$-th state at the next time $t+1$. Using the conditional probability, we can derive the relation between the existence probability $p_{i}(t)$ and the transition probability $T_{ij}$ as
\begin{align}
\tilde{p}_{ij}(t + 1) = T_{ij} p_{j}(t)\quad\text{for $1\leq i,j\leq 2^{N}$}.
\end{align}
From a property as the probability, it should hold that $\sum_{i=1}^{2^{N}}p_{i}(t)=1$ and $p_{i}(t) \ge 0$ ($i=1,2,\dots,2^{N}, t\in\mathbb{R}$). The conditional probability $\tilde{p}_{ij}(t)$ should also satisfy the condition that $\sum_{j=1}^{2^{N}}\tilde{p}_{ij}(t) = p_{i}(t+1)$ ($i=1,2,\dots,2^{N}, t\in\mathbb{R}$). Then we have
\begin{align}
p_{i}(t+1) = \sum_{j=1}^{2^{N}}\tilde{p}_{ij}(t + 1) = \sum_{j=1}^{2^{N}}T_{ij}p_{j}(t)\quad\text{for $1\leq i\leq 2^{N}$, $t\in\mathbb{R}$}.
\end{align}
In the other words, the system can be described by the probability vector $\bm{p}(t):={}^{\rm t}\left(p_{1}(t),p_{2}(t),\dots,p_{2^{N}}(t)\right)$ and the stochastic matrix $\hat{T}:=\left(T_{ij}\right)$ as
\begin{align}
\bm{p}(t + 1) = \hat{T}\bm{p}(t)\quad\text{for $t\in\mathbb{R}$}.
\end{align}
In Monte Carlo simulations, we often trace the trajectory of a component of the vector $\bm{p}(t)$, thus we rarely need to construct the matrix $\hat{T}$. But it is helpful for us to consider such the matrix when we discuss the convergence to the stationary state or its uniqueness. These discussions are valid for general $\Omega$-dimensional state spaces, thus we denote the number of states by $\Omega$ from now on.

\section{General Theory of Stochastic Matrices}

In this section we discuss the conditions which ensure a convergence of the corresponding Monte Carlo simulation to a unique stationary state. We first define the stochastic matrix and discuss fundamental properties of the stochastic matrix. We next discuss the properties which result from an additional condition called \textit{weak/strong connectivity}, which leads the existence and the uniqueness of a stationary state. We also see that we can construct a stochastic matrix which leads to any desired stationary state under the so-called \textit{detailed balanced condition}.

From the condition $\sum_{i=1}^{\Omega}p_{i}(t)=1$ and $p_{i}(t) \ge 0$ ($i=1,2,\dots,\Omega, t\in\mathbb{R}$), we have a set of properties $\sum_{i=1}^{\Omega}T_{ij} = 1$, $T_{ij}\ge 0$ ($1\leq i\leq \Omega$). Any matrix with these conditions is called \textit{stochastic matrix} and shows the following interesting property:
\begin{theorem}
	Let $\hat{T}$ be a stochastic matrix, then all absolute values of eigenvalue are less than or equal to $1$. For any eigenvector $\bm{x}={}^{\rm t}\{x_{1},x_{2},\dots,x_{\Omega}\}$ which does \textit{not} belong to the eigenvalue $1$, it additionally holds that
	\begin{align}
	\sum_{i=1}^{\Omega}x_{j}=0.
	\end{align}
\end{theorem}
We now define the vector $\bm{d}:={}^{\rm t}\left(1,1,\dots,1\right)$ to prove all facts after this.
\begin{proof}
	For any stochastic matrix $\hat{T}$, we have
	\begin{align}
	&\left({}^{\rm t}\hat{T}\bm{d}\right)_{i}=\sum_{j=1}^{N}\left({}^{t}T\right)_{ij}d_{j}=\sum_{j=1}^{N}T_{ji}d_{j}=\sum_{j=1}^{N}T_{ji}=1\quad\text{for $i=1,2,\dots,N$},\\
	\Longleftrightarrow\quad &{}^{\rm t}\hat{T}\bm{d} = \bm{d}.
	\end{align}
	Therefore the matrix ${}^{\rm t}\hat{T}$ has an eigenvalue $1$ at least. The eigenequation for the matrix ${}^{\rm t}\hat{T}$ are rewritten as
	\begin{align}
	\det\left[\lambda \hat{I}_{N} - {}^{\rm t}\hat{T}\right] = \det\left[{}^{\rm t}\left(\lambda \hat{I}_{N} - \hat{T}\right)\right] = \det\left[\lambda \hat{I}_{N} - \hat{T}\right],
	\end{align}
	and then the set of eigenvalues of $\hat{T}$ is equal to that of ${}^{\rm t}\hat{T}$. Finally the matrix $\hat{T}$ has an eigenvalue $1$ at least. A general eigenvalue equation of $\hat{T}$ can be written as
	\begin{align}
	\hat{T}\bm{x}_{\lambda} = \lambda\bm{x}_{\lambda}\label{eq:GenEigT},
	\end{align}
	where $\bm{x}_{\lambda}={}^{\rm t}\left(x_{\lambda,1},x_{\lambda,2},\dots,x_{\lambda,N}\right)$ is its eigenvector. We have
	\begin{align}
	&\left((\text{\textit{l.h.s} of \ref{eq:GenEigT}}),\bm{d}\right) = (\hat{T}\bm{x}_{\lambda},\bm{d}) = (\bm{x}_{\lambda},{}^{\rm t}\hat{T}\bm{d}) = (\bm{x}_{\lambda},\bm{d}),\\
	&\left((\text{\textit{r.h.s} of \ref{eq:GenEigT}}),\bm{d}\right) = (\lambda\bm{x}_{\lambda},\bm{d}) = \lambda(\bm{x}_{\lambda},\bm{d}).\\
	\Longleftrightarrow \quad& (1-\lambda)(\bm{x}_{\lambda},\bm{d}) = 0\quad\Longleftrightarrow \quad \lambda = 1\text{ or }(\bm{x}_{\lambda},\bm{d}) = 0.\\
	\Longleftrightarrow \quad& \sum_{i=1}^{N}x_{\lambda,i} = 0\quad\text{if $\lambda \neq 1$}.
	\end{align}
	We additionally define the vector $\bm{y}_{\lambda}:={}^{\rm t}\left(|x_{\lambda,1}|,|x_{\lambda,2}|,\dots,|x_{\lambda,N}|\right)$ for any $\lambda$. From the equation $\sum_{j=1}^{N}T_{ij}x_{\lambda,j}=\lambda x_{i}(i=1,2,\dots,N)$ we have
	\begin{align}
	|\sum_{j=1}^{N}T_{ij}x_{\lambda,j}| &\leq \sum_{j=1}^{N}T_{ij}|x_{\lambda,j}|\quad(\because T_{ij}\geq 0\quad\text{for $j=1,2,\dots,N$})\\
	&=\left(\hat{T}\bm{y}_{\lambda}\right)_{i}\quad\text{\text{for $i=1,2,\dots,N$}}\label{ineq:Ty}.
	\end{align}
	and the left hand side of \eqref{ineq:Ty} are rewritten as
	\begin{align}
	|\sum_{j=1}^{N}T_{ij}x_{\lambda,j}| = |\lambda x_{\lambda,j}| = |\lambda|\times |x_{\lambda,j}| = |\lambda|\times \left(\bm{y}_{\lambda}\right)_{j},
	\end{align}
	thus we have
	\begin{align}
	& |\lambda|\times \left(\bm{y}_{\lambda}\right)_{j} \leq \left(\hat{T}\bm{y}_{\lambda}\right)_{i},\\
	\Longleftrightarrow \quad& |\lambda|\times \left(\bm{y}_{\lambda},\bm{d}\right) \leq \left(\hat{T}\bm{y}_{\lambda},\bm{d}\right) = \left(\bm{y}_{\lambda},{}^{\rm t}\hat{T}\bm{d}\right) = \left(\bm{y}_{\lambda},\bm{d}\right), \\
	\Longleftrightarrow \quad& |\lambda| \leq 1.
	\end{align}
\end{proof}

We limit the class of stochastic matrices to that of weakly connected ones from now on.
\begin{definition}
	For an arbitrary $1\leq i,j\leq N$, if there exists an $n(i,j)>0$ such that
	\begin{align}
	\left(\hat{T}^{n(i,j)}\right)_{ij}>0,
	\end{align}
	the matrix $\hat{T}$ is called \textit{weakly connected}. Note that for any $n'>n(i,j)$ it does \textit{not} follows that $\left(\hat{T}^{n'}\right)>0$.
\end{definition}

To make proofs easier, we also define the matrix $\hat{\mathcal{T}}_{\epsilon}$ ($\epsilon>0$) and discuss its properties. Denoting the maximum value of $n(i,j)$ by $\displaystyle n_{\rm max}:=\max_{1\leq i,j\leq N}\left[n(i,j)\right]$ and defining the matrix $\hat{\mathcal{T}}_{\epsilon}:=\left(\hat{I}_{N}+\epsilon \hat{T}\right)^{n_{\rm max}}$, we have
	\begin{align}
	\left(\hat{\mathcal{T}}_{\epsilon}\right)_{ij} =& \left(\left(\hat{I}_{N}+\epsilon \hat{T}\right)^{n_{\rm max}}\right)_{ij} = \sum_{k=1}^{n_{\rm max}}\binom{n_{\rm max}}{k}\left({\hat{I}_{N}}^{k}\left(\epsilon \hat{T}\right)^{n_{\rm max}-k}\right)_{ij}\\
	=& \sum_{k=1}^{n_{\rm max}}\binom{n_{\rm max}}{k}\epsilon^{n_{\rm max}-k}\left(\hat{T}^{n_{\rm max}-k}\right)_{ij} \geq 0 \quad(\because T_{ij}>0)\quad\text{for $1\leq i,j\leq N$}.
	\end{align}

For the eigenvector $\bm{x}_{1}={}^{\rm t}\left(x_{1,1},x_{1,2},\dots,x_{1,N}\right)$, which belongs to the eigenvalue $1$, it holds that
\begin{align}
\hat{\mathcal{T}}_{\epsilon}\bm{x}_{1} &= \sum_{k=1}^{n_{\rm max}}\binom{k}{n_{\rm max}}\epsilon^{n_{\rm max}-k}\hat{T}^{n_{\rm max}}\bm{x}_{1}\\
&= \sum_{k=1}^{n_{\rm max}}\binom{k}{n_{\rm max}}\epsilon^{n_{\rm max}-k}\bm{x}_{1}\\
&= (1+\epsilon)^{n_{\rm max}}\bm{x}_{1},
\end{align}
and each component is
\begin{align}
\sum_{j=1}^{N}\left(\hat{\mathcal{T}}_{\epsilon}\right)_{ij}x_{1,j} = (1+\epsilon)^{n_{\rm max}}x_{1,i}\quad\text{for $i=1,2,\dots,N$}\label{eq:EigEqTcal}.
\end{align}

\begin{theorem}
	The phases of components of the vector $\bm{x}_{1}$ are aligned together and all the components are positive. In other words, we can decompose the vector into a phase factor and a positive vector as follows
	\begin{align}
	\bm{x}_{1} = \mathrm{e}^{i\theta}\bm{u}_{1},
	\end{align}
	where $\theta$ is the phase and $\bm{u}_{1}$ is the vector with all positive component.
\end{theorem}

\begin{proof}
	If components of the vector $\bm{x}_{1}$ are \textit{not} aligned together such that $\sum_{i=1}^{N}|x_{1,i}|>|\sum_{i=1}^{N}x_{1,i}|$ holds, we have
	\begin{align}
	|\sum_{j=1}^{N}\left(\hat{\mathcal{T}}_{\epsilon}\right)_{ij}x_{1,j}| < \sum_{j=1}^{N}\left(\hat{\mathcal{T}}_{\epsilon}\right)_{ij}|x_{1,j}| = (1+\epsilon)^{n_{\rm max}}|x_{1,i}|.
	\end{align}
	On the other hand, the row-wise sum of the matrix $\hat{\mathcal{T}}_{\epsilon}$ are
	\begin{align}
	\sum_{i=1}^{N}\left(\hat{\mathcal{T}}_{\epsilon}\right)_{ij} = \sum_{k=1}^{n_{\rm max}}\binom{k}{n_{\rm max}}\epsilon^{n_{\rm max}-k}\sum_{i=1}^{N}\left(\hat{T}^{n_{\rm max}-k}\right)_{ij} = (1+\epsilon)^{n_{\rm max}}.
	\end{align}
	Then we have
	\begin{align}
	\sum_{i=1}^{N}\sum_{j=1}^{N}\left(\hat{\mathcal{T}}_{\epsilon}\right)_{ij}|x_{1,j}| = (1+\epsilon)^{n_{\rm max}}\sum_{j=1}^{N}|x_{1,j}| > (1+\epsilon)^{n_{\rm max}}\sum_{i=1}^{N}|x_{1,i}|,
	\end{align}
	but it is the contradiction caused from our assumption $\sum_{i=1}^{N}|x_{1,i}|>|\sum_{i=1}^{N}x_{1,i}|$. Furthermore the left hand side of \eqref{eq:EigEqTcal} is positive because that $n_{\rm max}$ is the maximum value of $n(i,j)$, and then the right hand side is also positive. Then we have $x_{1,i}>0(i=1,2,\dots,N)$.
\end{proof}

\begin{theorem}
	The eigenspace of the matrix $\hat{\mathcal{T}}_{\epsilon}$, which belongs to the eigenvalue $1$, is \textit{one-dimensional}. 
\end{theorem}

\begin{proof}
	If we have two different eigenvectors, which belongs to the eigenvalue $1$, we can write their eigenequations by two different \textit{positive vectors} as
	\begin{align}
	\hat{T}\bm{u}_{1} = \bm{u}_{1},\\
	\hat{T}\bm{v}_{1} = \bm{v}_{1}.
	\end{align}
	For their any linear superposition, we also have
	\begin{align}
	\hat{T}(\bm{u}_{1} + t{v}_{1}) = \bm{u}_{1} + t{v}_{1},\quad\text{for any $t\in\mathbb{R}$}.
	\end{align}
	But if two eigenvectors $\bm{u}_{1}$ and $\bm{v}_{1}$ are not aligned, we can make a non-trivial vector with a certain $t$ such that $\left(\bm{u}_{1} + t{v}_{1}\right)_{l} = 0$ for an $l$-th element. But it is the contradiction with the fact $x_{1,i}>0(i=1,2,\dots,N)$. Then we have no eigenspaces more than one, which belongs to the eigenvalue $1$.
\end{proof}

We additionally limit the class of stochastic matrices to that of strongly connected ones from now on.
\begin{definition}
	If there exists a number $N_{0}>0$ such that
	\begin{align}
	\left(\hat{T}^{N_{0}}\right)_{ij}>0
	\end{align}
	for an arbitrary $1\leq i,j\leq N$, the matrix $\hat{T}$ is called \textit{strongly connected}.
\end{definition}

\begin{theorem}
	There exists only the eigenvalue $1$ with its absolute value $1$.
\end{theorem}

\begin{proof}
	We now have $\hat{T}^{n}\bm{u}_{\lambda}=\lambda^{n}\bm{u}_{\lambda}$, where $\bm{u}_{\lambda} = {}^{\rm t}\left(u_{\lambda,1}, u_{\lambda,2}, \dots, u_{\lambda,N}\right)$ is the eigenvector which belongs to an eigenvalue $\lambda$. Their components are written as
	\begin{align}
	\sum_{j=1}^{N}\left(\hat{T}^{n}\right)_{ij}u_{\lambda,j} = \lambda^{n} \bm{u}_{\lambda,i},\quad\text{for $i = 1,2,\dots,N$}.
	\end{align}
	We can divide conditions for $\lambda$ into following two cases:
	\begin{description}
		\item[Case1: $\sum_{i=1}^{N}|u_{\lambda,i}| > |\sum_{i=1}^{N}u_{\lambda,i}|$,]\mbox{}\\
		We have
		\begin{align}
		&\sum_{j=1}^{N}\left(\hat{T}^{n}\right)_{ij}|u_{\lambda,j}| > |\sum_{j=1}^{N}\left(\hat{T}^{n}\right)_{ij}u_{\lambda,j}| = |\lambda^{n}|\times|u_{\lambda,i}|,\quad\text{for $i = 1,2,\dots,N$}.\\
		\Longleftrightarrow\quad & |\lambda^{n}| < 1 \quad \Longleftrightarrow\quad |\lambda| < 1.
		\end{align}
		\item[Case2: $\sum_{i=1}^{N}|u_{\lambda,i}| = |\sum_{i=1}^{N}u_{\lambda,i}|$.]\mbox{}\\
		We have
		\begin{align}
		& \sum_{i=1}^{N}\sum_{j=1}^{N}\left(\hat{T}^{n}\right)_{ij}u_{\lambda,j}  = \sum_{j=1}^{N}u_{\lambda,j}  = \lambda^{n}\sum_{i=1}u_{\lambda,i}.\\
		\Longleftrightarrow\quad & \lambda^{n} = 1\quad(\because \bm{u}_{\lambda}\neq\bm{0},u_{\lambda,i}\geq 0 \Rightarrow \sum_{i=1}^{N}u_{\lambda,i} > 0).
		\end{align}
	\end{description}
	Thus there is only an eigenvalue $1$ with its absolute value $1$.
\end{proof}

\begin{theorem}
	The vector $\lim_{N\to\infty}\hat{T}^{N}\bm{r} = \bm{0}$ for any $\bm{r}\in\mathbb{C}$ is orthogonal to $\bm{d}$.
\end{theorem}

\begin{proof}
	For an arbitrary vector $\bm{r}$, we can decompose it into its real and imaginary parts as $\bm{r}=\bm{r}_{\rm R} + i\bm{r}_{\rm I}$. Since the condition $(\bm{r},\bm{d}) = 0$ is equivalent to $\sum_{i=1}^{N}r_{i} = 0$, we have
	\begin{align}
	\sum_{j\in I_{+}}r_{j} + \sum_{j\in I_{-}}r_{j} = 0,
	\end{align}
	where $I_{\pm}:=\left\{j\mid r_{j}\gtrless 0,1\leq j\leq N\right\}$. Note that $\sum_{j\in I_{+}}r_{j} = \sum_{j\in I_{-}}|r_{j}|$.  Thus we have
	\begin{align}
	\sum_{j\in I_{+}}r_{j} = \sum_{j\in I_{-}}|r_{j}| = \|\bm{r}\|_{1} / 2.
	\end{align}
	Since $\hat{T}$ is strongly connected, there is an integer $N_{0}$ such that $\left(\hat{T}^{N_{0}}\right)_{ij}>0$ for an arbitrary $1\leq i,j\leq N$. For the $N_{0}$ we have
	\begin{align}
	\left(\hat{T}^{N_{0}}\bm{r}\right) = \sum_{j=1}^{N}\left(\hat{T}^{N_{0}}\right)_{ij}r_{j} =& \sum_{j\in I_{+}}\left(\hat{T}^{N_{0}}\right)_{ij}r_{j} - \sum_{j\in I_{-}}\left(\hat{T}^{N_{0}}\right)_{ij}|r_{j}|\\
	=& \sum_{j=1}^{N}\left(\hat{T}^{N_{0}}\right)_{ij}r_{j} - 2\sum_{j\in I_{-}}\left(\hat{T}^{N_{0}}\right)_{ij}|r_{j}|\\
	\leq& \sum_{j=1}^{N}\left(\hat{T}^{N_{0}}\right)_{ij}r_{j} - 2\delta_{N_{0}}\sum_{j\in I_{-}}|r_{j}|\\
	=& \sum_{j=1}^{N}\left(\hat{T}^{N_{0}}\right)_{ij}r_{j} - \delta_{N_{0}}\|\bm{r}\|_{1},
	\end{align}
	where $\displaystyle\delta_{N_{0}}:= \min_{1\leq i,j\leq N}\left[\left(\hat{T}^{N_{0}}\right)_{ij}\right]$ (for $i=1,2,\dots,N$). Note that there exists a $\delta_{N_{0}} > 0$ for the strongly connected matrix   $\hat{T}$. Similarly we have
	\begin{align}
	\left(\hat{T}^{N_{0}}\bm{r}\right) = \sum_{j=1}^{N}\left(\hat{T}^{N_{0}}\right)_{ij}r_{j} =& \sum_{j\in I_{+}}\left(\hat{T}^{N_{0}}\right)_{ij}r_{j} - \sum_{j\in I_{-}}\left(\hat{T}^{N_{0}}\right)_{ij}|r_{j}|\\
	=& 2\sum_{j\in I_{+}}\left(\hat{T}^{N_{0}}\right)_{ij}r_{j} - \sum_{j=1}^{N}\left(\hat{T}^{N_{0}}\right)_{ij}|r_{j}|\\
	\geq& 2\delta_{N_{0}}\sum_{j\in I_{+}}r_{j} - \sum_{j=1}^{N}\left(\hat{T}^{N_{0}}\right)_{ij}|r_{j}|\\
	=& \delta_{N_{0}}\|\bm{r}\|_{1} - \sum_{j=1}^{N}\left(\hat{T}^{N_{0}}\right)_{ij}|r_{j}|,\quad\text{(for $i=1,2,\dots,N$)}.
	\end{align}
	Combining them, we have
	\begin{align}
	|\left(\hat{T}^{N_{0}}\bm{r}\right)| \leq \sum_{j=1}^{N}\left(\hat{T}^{N_{0}}\right)_{ij}|r_{j}| - \delta_{N_{0}}\|\bm{r}\|_{1},\quad\text{(for $i=1,2,\dots,N$)},
	\end{align}
	and then it holds that
	\begin{align}
	\|\hat{T}^{N_{0}}\bm{r}\|_{1} = \sum_{i=1}^{N}|\left(\hat{T}^{N_{0}}\bm{r}\right)_{i}| \leq \sum_{j=1}^{N}|r_{j}| - N_{0}\delta_{N_{0}}\|\bm{r}\|_{1} = (1-N\delta_{N_{0}})\|\bm{r}\|_{1}.
	\end{align}
	The vector $\hat{T}^{N_{0}}\bm{r}$ is also orthogonal to the vector $\bm{d}$, actually it holds that
	\begin{align}
	\left(\hat{T}^{N_{0}}\bm{r},\bm{d}\right) = \left(\bm{r},{}^{\rm t}\left(\hat{T}^{N_{0}}\right)\bm{d}\right) = \left(\bm{r},\left({}^{\rm t}\hat{T}\right)^{N_{0}}\bm{d}\right) = \left(\bm{r},\bm{d}\right) = 0.
	\end{align}
	Then, for any positive integer $l$, we can repeat this discussion as
	\begin{align}
	\|\hat{T}^{N_{0}l}\bm{r}\|_{1} \leq (1-N_{0}\delta_{N_{0}})^{l}\|\bm{r}\|_{1}.
	\end{align}
	Since $N_{0}>0, \delta_{N_{0}}>0$ and thus $1-N_{0}\delta_{N_{0}} < 0$, we have
	\begin{align}
	&\lim_{l\to\infty}(1-N_{0}\delta_{N_{0}})^{l}\|\bm{r}\|_{1} = 0.\\
	\Longleftrightarrow\quad&\lim_{l\to\infty}\|\hat{T}^{N_{0}l}\bm{r}\|_{1} = 0.
	\end{align}
	Thus, for an arbitrary positive integer $N$, we have
	\begin{align}
	\lim_{N\to\infty}\|\hat{T}^{N}\bm{r}\|_{1} = 0.
	\end{align}
\end{proof}

\begin{theorem}\label{theo:SupPos}
	We can write any vector $\bm{x}$ as the superposition of $\bm{u}_{1}$ and $\bm{r}$.
\end{theorem}

\begin{proof}
	Defining the coefficient $c_{1,\bm{x}}:=(\bm{x},\bm{d})/(\bm{u}_{1},\bm{d})$ and the vector $\bm{r}_{\bm{x}}:=\bm{x} - c_{1,\bm{x}}\bm{u}_{1}$, we have
	\begin{align}
	(\bm{r}_{\bm{x}},\bm{d}) &= (\bm{x},\bm{d}) - \frac{(\bm{x},\bm{d})}{(\bm{u}_{1},\bm{d})}(\bm{u}_{1},\bm{d}) = 0,\\
	\bm{x} &= c_{1,\bm{x}}(\bm{u}_{1},\bm{d}) + \bm{r}_{\bm{x}}.
	\end{align}
\end{proof}

\begin{theorem}
	The limit $\lim_{N\to\infty}\hat{T}^{N}\bm{p}^{(0)}$ is independent on the initial vector $\bm{p}^{(0)}$ and it holds that
	\begin{align}
	\lim_{N\to\infty}\hat{T}^{N}\bm{p}^{(0)} = \frac{\bm{u}_{1}}{\|\bm{u}_{1}\|_{1}},
	\end{align}
	where the vector $\bm{p}^{(0)}={}^{\rm t}\left\{p^{(0)}_{1}, p^{(0)}_{2}, \dots, p^{(0)}_{\Omega}\right\}$ is in the class of probability vectors and then it is normalized $\sum_{i=1}^{\Omega}p^{(0)}_{i} = 1$.
\end{theorem}

\begin{proof}
	From the theorem \ref{theo:SupPos}, we have
	\begin{align}
	\hat{T}^{N}\bm{p}^{(0)} = c_{1,\bm{p}^{(0)}}\hat{T}^{N}\bm{u}_{1} + \hat{T}^{N}\bm{r}_{\bm{p}^{(0)}} = c_{1,\bm{p}^{(0)}}\bm{u}_{1} + \hat{T}^{N}\bm{r}_{\bm{p}^{(0)}}.
	\end{align}
	Its limit $N\to\infty$ is taken as
	\begin{align}
	\lim_{N\to\infty}\hat{T}^{N}\bm{p}^{(0)} = c_{1,\bm{p}^{(0)}}\bm{u}_{1} + \lim_{N\to\infty}\hat{T}^{N}\bm{r}_{\bm{p}^{(0)}} = c_{1,\bm{p}^{(0)}}\bm{u}_{1}.
	\end{align}
	Using the matrix $\hat{A}:=\bm{u}_{1}{}^{\rm t}\bm{d}/(\bm{u}_{1},\bm{d})$, we have
	\begin{align}
	\left(\hat{A}\bm{p}^{(0)}\right)_{i} = \sum_{j = 1}^{N}\frac{\left(\bm{u}_{1}\right)_{i}}{(\bm{u}_{1},\bm{d})}\left(\bm{p}^{(0)}\right)_{j} = \left(\bm{p}^{(0)},\bm{d}\right)\frac{\left(\bm{u}_{1}\right)_{i}}{(\bm{u}_{1},\bm{d})},\quad\text{for $i=1,2,\dots,N$}.
	\end{align}
	Then it leads
	\begin{align}
	& \hat{A}\bm{p}^{(0)} = \frac{\left(\bm{u}_{1}\right)_{i}}{(\bm{u}_{1},\bm{d})}\bm{p}^{(0)} = c_{1,\bm{p}^{(0)}}\bm{p}^{(0)}.
	\end{align}
	Then the limit $N\to\infty$ for $\hat{T}^{N}\bm{p}^{(0)}$ is rewritten by a simple multiplication as follows
	\begin{align}
	\lim_{N\to\infty}\hat{T}^{N}\bm{p}^{(0)} = \frac{\bm{u}_{1}{}^{\rm t}\bm{d}}{(\bm{u}_{1},\bm{d})}\bm{p}^{(0)}.\label{for:TNP}
	\end{align}
	The expression \eqref{for:TNP} lead to the relation $\lim_{N\to\infty}\hat{T}^{N}\bm{p}^{(0)} = \bm{u}_{1}/\|\bm{u}_{1}\|_{1}$. Actually it holds that
	\begin{align}
	\frac{\bm{u}_{1}{}^{\rm t}\bm{d}}{(\bm{u}_{1},\bm{d})}\left(\bm{p}^{(0)}\right)_{i} = \frac{\sum_{j=1}^{N}\left(\bm{u}_{1}\right)_{i}\left(\bm{d}\right)_{j}\left(\bm{p}^{(0)}\right)_{j}}{(\bm{u}_{1},\bm{d})} = \frac{u_{1,i}}{\|\bm{u}_{1}\|_{1}},
	\end{align}
	thus we have
	\begin{align}
	\lim_{N\to\infty}\hat{T}^{N}\bm{p}^{(0)} = \frac{\bm{u}_{1}}{\|\bm{u}_{1}\|_{1}}.
	\end{align}
\end{proof}

Once we get a strongly connected stochastic matrix $\hat{T}$, its stationary distribution $\bm{u}_{1}/\|\bm{u}_{1}\|_{1}$ is determined independently on the initial distribution. We can consider this procedure as a \textit{transformation} from the matrix $\hat{T}$ into the vector $\bm{u}_{1}/\|\bm{u}_{1}\|_{1}$. 

On the other hand, we can also consider the inverse transformation. This means the \textit{construction} of the matrix $\hat{T}'$ with a \textit{desired} stationary distribution $\bm{p}'$. We can actually formulate such the procedure using the \textit{detailed balanced condition} as follows.
\begin{itembox}{Detailed Balanced Condition}
If a vector $\bm{p}'={}^{\rm t}(p'_{1},p'_{2},\dots,p'_{\Omega})$ is the stationary distribution of the stochastic matrix $\hat{T}'=\left(T'_{ij}\right)$, it holds that
\begin{align}
T'_{ij}p'_{j} = T'_{ji}p'_{i}\label{cond:D.B.C}
\end{align}
for $1\leq i,j\leq \Omega$.
\end{itembox}
This property is simplified by summing over the subscription $i$ as follows
\begin{align}
\sum_{i=1}^{\Omega}T'_{ij}p'_{j} = p'_{j} = \sum_{i=1}^{\Omega}T'_{ji}p'_{i} = \left(\hat{T}\bm{p}'\right)_{j},\quad\text{for $j=1,2,\dots,\Omega$}.
\end{align}
Then if the detailed balanced condition holds for a matrix $\hat{T}'$, the corresponding vector $\bm{p}'$ is the fixed point of the matrix $\hat{T}'$. Thus in order to get the matrix $\hat{T}'$ which lead any distribution $\bm{p}^{(0)}$ to the desired distribution $\bm{p}'$, we only have to construct each element $T'_{ij}$ of the matrix according to the condition \eqref{cond:D.B.C} and make the matrix strongly connected.

In Monte Carlo simulations of statistical mechanics, we calculate the equilibrium distribution $\bm{p}_{\rm eq}(\beta)$ at an inverse temperature $\beta$ using an initial state $i_{0}$ and a rule of the stochastic process $i\to j\to j'\to\dots$. We can regard the matrix construction method as a set of the stochastic process $i_{0}\to j\to j'\to\dots$ over all possible states $i_{0}=1,2,\dots,\Omega$. In other words, the matrix $T_{ij}$ corresponding to a simulation contains the information of all possible initial states $i_{0}=1,2,\dots,\Omega$ and all possible transitions $T_{1,i_{0}},T_{2,i_{0}},\dots,T_{\Omega,i_{0}}$ respectively from them. Thus we often consider a \textit{sample path} by an appropriate initial state and the matrix.

The condition \eqref{cond:D.B.C} or
\begin{align}
\frac{T_{ij}}{T_{ji}} = \frac{p_{i}}{p_{j}}
\end{align}
is not sufficient to determine a concrete form of the matrix $\hat{T}$ and thus in general there is a degree of freedom in the form of $\hat{T}$. For Monte Carlo simulations of equilibrium statistical mechanics, this freedom also remains as follows:
\begin{align}
\frac{T_{ij}(\beta)}{T_{ji}(\beta)} = \frac{p_{i}(\beta)}{p_{j}(\beta)} = \mathrm{e}^{-\beta (E_{i}-E_{j})},\quad\text{for $1\leq i,j\leq N$},
\end{align}
where ${T_{ij}(\beta)}$ and $p_{i}(\beta)$ are the matrix corresponding to a simulation and the probability that the $i$-th state emerges respectively with a fixed inverse temperature $\beta$. But this relation is important that the rate of transition probabilities is always given by a difference of energy eigenvalues $\left\{E_{1},E_{2},\dots,E_{\Omega}\right\}$, and thus we do \textit{not} have to calculate exact values of $\tilde{p}_{i}(\beta):=\exp\left[-\beta E_{i}\right]/Z(\beta)$ ($Z(\beta):=\sum_{i}\exp\left[-\beta E_{i}\right]$).

If we use the Metropolis probability for the $N$-spins system, each element of the matrix $\left(T_{ij}(\beta)\right)$ is written as follows.
\begin{align}
T_{ij}(\beta)=
\begin{cases}
\frac{1}{N}\min\left[1,\mathrm{e}^{-\beta (E_{i}-E_{j})}\right]\quad&\text{for states $i,j$ mutually reacheable by a single flip},\\
0\quad&\text{for states $i,j$ \textit{not} mutually reacheable by a single flip}.
\end{cases}
\end{align}
The factor $1/N$ describes the equivalent selection of a spin to flip.

\subsection{Construction of the Stochastic Matrix based on the Detailed Balanced Condition}

\textcolor{red}{Cite the mathematica results}.

\section{Stochastic Matrices and Non-Equilbrium Monte Carlo Simulations}

\section{Calculation Non-Equilbrium Observables by the Stochastic Matrices}
