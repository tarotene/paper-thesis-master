\documentclass[12pt]{book}
% \usepackage[english]{babel}

%\topmargin=20mm
%\headheight=12pt
%\headsep=0mm
%\textheight=200mm
%\oddsidemargin=5mm
%\evensidemargin=5mm
%\textwidth=150mm

\topmargin=-10mm
\headheight=12pt
\headsep=0mm
\textheight=245mm
\oddsidemargin=-5mm
\evensidemargin=-5mm
\textwidth=170mm

\usepackage{amsmath,amsfonts,amsthm,amssymb}
\usepackage{graphicx}
\usepackage{color}
\usepackage{subfigure}
\usepackage{cite}
\usepackage{dcolumn}
\usepackage{bm}
\usepackage{book tabs}
\usepackage{braket}
\usepackage{setspace}
% \usepackage{docmute}

\newcounter{one}
\setcounter{one}{1}

\pagestyle{plain}

% \newcommand{\kB}{{k_\mathrm{B}}}
% \newcommand{\TR}{{T_\mathrm{R}}}
% \newcommand{\TL}{{T_\mathrm{L}}}
% \newcommand{\QR}{{Q_\mathrm{R}}}
% \newcommand{\QL}{{Q_\mathrm{L}}}
% \newcommand{\UR}{{U_\mathrm{R}}}
% \newcommand{\UL}{{U_\mathrm{L}}}
% \newcommand{\WR}{{W_\mathrm{R}}}
% \newcommand{\WL}{{W_\mathrm{L}}}
% \newcommand{\NR}{{N_\mathrm{R}}}
% \newcommand{\NL}{{N_\mathrm{L}}}
% \newcommand{\betaR}{{\beta_\mathrm{R}}}
% \newcommand{\betaL}{{\beta_\mathrm{L}}}
% \newcommand{\fR}{{f_\mathrm{R}}}
% \newcommand{\fL}{{f_\mathrm{L}}}
% \newcommand{\chicR}{{\chi_{c\mathrm{R}}}}
% \newcommand{\chicL}{{\chi_{c\mathrm{L}}}}
% \newcommand{\chihR}{{\chi_{h\mathrm{R}}}}
% \newcommand{\chihL}{{\chi_{h\mathrm{L}}}}
% \newcommand{\chiQL}{{\chi_{Q}^{\mathrm{L}}}}
% \newcommand{\chiQR}{{\chi_{Q}^{\mathrm{R}}}}
% \newcommand{\chiNL}{{\chi_{N}^{\mathrm{L}}}}
% \newcommand{\chiNR}{{\chi_{N}^{\mathrm{R}}}}
% \newcommand{\chiqR}{{\chi_{q}^{\mathrm{R}}}}
% \newcommand{\chiqL}{{\chi_{q}^{\mathrm{L}}}}
% \newcommand{\chiQLL}{{\chi_{Q\mathrm{L}}^{\mathrm{L}}}}
% \newcommand{\chiQLR}{{\chi_{Q\mathrm{R}}^{\mathrm{L}}}}
% \newcommand{\chiQRL}{{\chi_{Q\mathrm{L}}^{\mathrm{R}}}}
% \newcommand{\chiQRR}{{\chi_{Q\mathrm{R}}^{\mathrm{R}}}}
% \newcommand{\chiNLL}{{\chi_{N\mathrm{L}}^{\mathrm{L}}}}
% \newcommand{\chiNLR}{{\chi_{N\mathrm{R}}^{\mathrm{L}}}}
% \newcommand{\chiNRL}{{\chi_{N\mathrm{L}}^{\mathrm{R}}}}
% \newcommand{\chiNRR}{{\chi_{N\mathrm{R}}^{\mathrm{R}}}}
% \newcommand{\chiqRL}{{\chi_{q\mathrm{L}}^{\mathrm{R}}}}
% \newcommand{\chiqRR}{{\chi_{q\mathrm{R}}^{\mathrm{R}}}}
% \newcommand{\chiqLL}{{\chi_{q\mathrm{L}}^{\mathrm{L}}}}
% \newcommand{\chiqLR}{{\chi_{q\mathrm{R}}^{\mathrm{L}}}}
% \newcommand{\AcR}{{A_{c\mathrm{R}}}}
% \newcommand{\AcL}{{A_{c\mathrm{L}}}}
% \newcommand{\AhR}{{A_{h\mathrm{R}}}}
% \newcommand{\AhL}{{A_{h\mathrm{L}}}}
% \newcommand{\ANR}{{A_N^{\mathrm{R}}}}
% \newcommand{\ANL}{{A_N^{\mathrm{L}}}}
% \newcommand{\ANRL}{{A_{N\mathrm{L}}^{\mathrm{R}}}}
% \newcommand{\ANRR}{{A_{N\mathrm{R}}^{\mathrm{R}}}}
% \newcommand{\ANLL}{{A_{N\mathrm{L}}^{\mathrm{L}}}}
% \newcommand{\ANLR}{{A_{N\mathrm{R}}^{\mathrm{L}}}}
% \newcommand{\AQR}{{A_Q^{\mathrm{R}}}}
% \newcommand{\AQL}{{A_Q^{\mathrm{L}}}}
% \newcommand{\AQRL}{{A_{Q\mathrm{L}}^{\mathrm{R}}}}
% \newcommand{\AQRR}{{A_{Q\mathrm{R}}^{\mathrm{R}}}}
% \newcommand{\AQLR}{{A_{Q\mathrm{R}}^{\mathrm{L}}}}
% \newcommand{\AQLL}{{A_{Q\mathrm{L}}^{\mathrm{L}}}}
% \newcommand{\AqR}{{A_q^{\mathrm{R}}}}
% \newcommand{\AqL}{{A_q^{\mathrm{L}}}}
% \newcommand{\AqRL}{{A_{q\mathrm{L}}^{\mathrm{R}}}}
% \newcommand{\AqRR}{{A_{q\mathrm{R}}^{\mathrm{R}}}}
% \newcommand{\AqLR}{{A_{q\mathrm{R}}^{\mathrm{L}}}}
% \newcommand{\AqLL}{{A_{q\mathrm{L}}^{\mathrm{L}}}}
% \newcommand{\Ah}{{A_h}}
% \newcommand{\Ac}{{A_c}}
% \newcommand{\pR}{{p_\mathrm{R}}}
% \newcommand{\pL}{{p_\mathrm{L}}}
% \newcommand{\muR}{{\mu_\mathrm{R}}}
% \newcommand{\muL}{{\mu_\mathrm{L}}}
% \newcommand{\EG}{{E_\mathrm{G}}}
% \newcommand{\JQR}{{J_Q^{\mathrm{R}}}}
% \newcommand{\JQL}{{J_Q^{\mathrm{L}}}}
% \newcommand{\JNR}{{J_N^{\mathrm{R}}}}
% \newcommand{\JNL}{{J_N^{\mathrm{L}}}}
% \newcommand{\JOR}{{J_\Omega^{\mathrm{R}}}}
% \newcommand{\JOL}{{J_\Omega^{\mathrm{L}}}}
% \newcommand{\JQl}{{J_Q^{\ell}}}
% \newcommand{\JSR}{{J_S^{\mathrm{R}}}}
% \newcommand{\JSL}{{J_S^{\mathrm{L}}}}
% \newcommand{\JSl}{{J_S^{\ell}}}
% \newcommand{\sR}{{s_\mathrm{R}}}
% \newcommand{\sL}{{s_\mathrm{L}}}
% \newcommand{\SRdot}{{\dot{S}_\mathrm{R}}}
% \newcommand{\SLdot}{{\dot{S}_\mathrm{L}}}
% \newcommand{\QRdot}{{\dot{Q}_\mathrm{R}}}
% \newcommand{\QLdot}{{\dot{Q}_\mathrm{L}}}
% \newcommand{\NRdot}{{\dot{N}_\mathrm{R}}}
% \newcommand{\NLdot}{{\dot{N}_\mathrm{L}}}
% \newcommand{\URdot}{{\dot{U}_\mathrm{R}}}
% \newcommand{\ULdot}{{\dot{U}_\mathrm{L}}}
% \newcommand{\WRdot}{{\dot{W}_\mathrm{R}}}
% \newcommand{\WLdot}{{\dot{W}_\mathrm{L}}}
% \newcommand{\dSRdot}{{\Delta \dot{S_\mathrm{R}}}}
% \newcommand{\dSLdot}{{\Delta \dot{S_\mathrm{L}}}}
% \newcommand{\Sdot}{{\dot{S}}}
% \newcommand{\Dl}{{D_{\ell}}}
% \newcommand{\DKR}{{D_{\mathrm{KR}}}}
% \newcommand{\llangle}{{\langle\! \langle}}
% \newcommand{\rrangle}{{\rangle\! \rangle}}
% \newcommand{\ain}{{a^{\text{in}}}}
% \newcommand{\aout}{{a^{\text{out}}}}
% \newcommand{\bin}{{b^{\text{in}}}}
% \newcommand{\bout}{{b^{\text{out}}}}
% \newcommand{\anin}{{a_n^{\text{in}}}}
% \newcommand{\anout}{{a_n^{\text{out}}}}
% \newcommand{\bnin}{{b_n^{\text{in}}}}
% \newcommand{\bnout}{{b_n^{\text{out}}}}
% \newcommand{\vecain}{{\bm{a}^{\text{in}}}}
% \newcommand{\vecaout}{{\bm{a}^{\text{out}}}}
% \newcommand{\vecbin}{{\bm{b}^{\text{in}}}}
% \newcommand{\vecbout}{{\bm{b}^{\text{out}}}}
% \newcommand{\SLL}{{S_\text{LL}}}
% \newcommand{\SLR}{{S_\text{LR}}}
% \newcommand{\SRL}{{S_\text{RL}}}
% \newcommand{\SRR}{{S_\text{RR}}}
% \newcommand{\SLa}{{S_{\text{L}\alpha}}}
% \newcommand{\SLb}{{S_{\text{L}\beta}}}
% \newcommand{\IL}{{I_{\mathrm{L}}}}
% \newcommand{\cre}[2]{{{#1}_{{#2}}^{\dagger}}}
% \newcommand{\ani}[2]{{{#1}_{{#2}}}}
% \newcommand{\dcre}[3]{{{#1}_{{#2},{#3}}^{\dagger}}}
% \newcommand{\dani}[3]{{{#1}_{{#2},{#3}}}}
% \newcommand{\dlcre}[3]{{{#1}_{\text{{#2}},{#3}}^{\dagger}}}
% \newcommand{\dlani}[3]{{{#1}_{\text{{#2}},{#3}}}}
% \newcommand{\dagg}[1]{{{#1}^\dagger}}
% \newcommand{\conjg}[1]{{{#1}^\ast}}
% \newcommand{\Tr}[1]{{\text{Tr}[{#1}]}}
% \newcommand{\expect}[1]{{\langle{#1}\rangle}}
% \newcommand{\cumulant}[1]{{\langle\!\langle{#1}\rangle\!\rangle}}
% \newcommand{\deriv}[2]{\frac{\text{d}{#1}}{\text{d}{#2}}}

%\renewcommand{\baselinestretch}{2} %ダブルスペース用
%\allowdisplaybreaks %数式の途中で改ページ許可
%\setstretch{1.1}

\begin{document}

\title{Effects of Boundary Conditions on Magnetic Friction}
\author{Kentaro Sugimoto \\ Department of Physics, The University of Tokyo }
\date{\today}
\maketitle

Acknowledgement.


\clearpage

\begin{center}
\large{{\bf Abstract}}
\end{center}

%%%%%
In the present thesis, hogehoge.
Moreover, fugafuga.
%%%%%

\tableofcontents

% !TeX root = Body.tex
\chapter{Introduction}
%The system is one of the simplest model of two-dimensional magnetic friction. Its spatial and spin dimensionality are far from realistic materials around us. However, we can use several facts from the exact solution for the two-dimensional Ising model, which makes the analysis easier than higher-dimensional cases.
The sliding friction in solids is a too complex problem to deal with, despite the fact that our daily lives are linked with it in various forms. One reason is that there is no general theory for various and a number of physical degrees of freedom, which determines the most important degree for the sliding friction as a phenomenon.

One may think that, with the skill of statistical mechanics, we can deal with the problem in a systematic manner. But there still remains several problems as follows:
\begin{itemize}
	\item \textbf{Problem1}: The sliding friction is essentially non-equilibrium phenomenon.
	\item \textbf{Problem2}: We cannot directly observe the sliding surface.
\end{itemize}

In this chapter we introduce two of the most famous problems with dealing the frictional force as the problem of statistical mechanics. We give recent developments for solving these problems. We then propose the question related to a problem about the \textit{manipulation} of the sliding friction which occurs in highly lubricated solids. To this end we simplify the problem into a dimensional crossover in lattice systems.

\section{Sliding Frictions as Non-Equilibrium Problems}
We can regard the sliding friction as follows in an elementary manner. We consider an object $O$ and a substrate $S$, and let $S$ slide against $O$ with them contacting and an external force. When $O$ and $S$ interact with each other, the kinetic energy of $O$ given by the external force is expected to lose by the interaction, and then the entire system $O+S$ heats up (if they form a closed system) or an energy dissipation occurs from the system to external environment (if they form an open system). In the latter case, under the assumption that the dissipation process stationarily occurs, the frictional force $f_{\rm fric}$ take a constant value balancing with the external force. This setup is realized when we keep the external force $f_{\rm ext}$ so that the sliding velocity $v$ take a constant value. Then the frictional force $f_{\rm fric}$ is dealt with as a function of the sliding velocity as $f_{\rm fric}=f_{\rm fric}(v)$.

The traditional way of statistical mechanics, called the linear response theory, appears to deal with the problem under the condition that the velocity $v$ is much smaller than the rate $\xi/\tau$, where $\xi$ and $\tau$ are the characteristic length and time of the system. But we already know well the phenomenon that the static frictional force is non-zero value for several systems. In such systems, we easily observe the non-linearity of the frictional force $f_{\rm fric}$ for the velocity $v$. This simply shows us the complexity of the problem which cannot be captured by applying the traditional way.

Many researches have dealt with the problem using the numerical way or limiting to an extreme region of parameters to avoid attack with the perturbative way.

\section{Impossibility of the Observation of the Sliding Surface}
The dimensionality of the sliding surface is up to two-dimension, if that of the whole system is the three-dimension. Sliding surfaces of such systems are different in many ways from well-known two-dimensional surfaces of three-dimensional solids which have investigated for many years, then we cannot perform a direct observation of the sliding surface by apparatuses such as microscopes. The difficulty prevents us from revealing non-equilibrium properties of the sliding friction.

There is the way to observe the contact plane with the microscope and an optically transparent matter, but most researches avoid a direct observations by measuring other observables to indirectly observe the contact plane.

\section{Manipulating the Friction}

Recent researches well revealed the nature of sliding frictions. This also leads us to conflict with a new problems about the friction in atomically microscopic systems.

Ordinary frictions in solids are mostly governed by excitations of phonon degrees of freedom, because the contact plane is almost always rough than the scale of the atom. But once we get the contact plane highly lubricated, other degrees of freedom, such as the orbital and the spin of electrons, emerges as the main contribution to the friction, in addition to the phonon excitation. 

We are already familiar with the most remarkable example of such a system in our daily lives, which is called \textit{micro electric mechanical system} (\textbf{MEMS}). MEMS plays an important role in the head of inkjet printers, the accelerometer in smartphones and so on. As an aspect of the MEMS, there are processed planes with an accuracy of a micrometer or a nanometer and they are also moving parts. Thus they inevitably experience the new type of the friction by operations. In addition the smaller size of these systems makes the problem more serious, because the rate of the surface area over the volume of a system become larger with the smaller size in general.

Thus we have to tackle with the issue of manipulation the friction in such a smaller system by getting more fundamental knowledges of the friction.

\section{Magnetic Friction}
The way to manipulate the friction in such a small system is less understood than its nature. Thus we consider the manipulation of magnetic materials as an easier problem to analyze by lattice models and its simulations. 

Frictions in such models themselves are a new type of the problem problems, and date back to the numerical research by Kadau et al.\cite{Kadau2008}. They have revealed that two square lattices of a Ising model which slide with each other experience the friction, depending on the temperature and the sliding velocity, by Monte Carlo simulations. Immediately after the research, it was revealed that the Ising model goes to a non-trivial non-equilibrium phase transitions (\textbf{NEPT}) in the high-velocity limit, where the two sliding models are decoupled in terms of the correlation between the two models and feel a mean field depending the magnetization of each other\cite{Hucht2009b}. By this treatment, we are able to access the novel critical point which is located in a higher temperature than the ordinal critical point in general for models with arbitrary dimensions and geometries (see Chapter \ref{ch:review}). In addition to the results, they developed a new algorithm which enables the analytical treatment in more detail and revealed the non-equilibrium critical point for arbitrary velocities. 

Based on their results\cite{Hucht2009b}, we consider a dimensional crossover from one-dimension to two in Ising models with two fixed boundary conditions. In the one-dimensional limit the boundary conditions seem to have the most effect on the friction, whereas in the two-dimensional limit there seems to be no effects. Behaviors in the both limits for the free boundary condition correspond to the results\cite{Hucht2009b}. In this way we think the problem of the manipulation as the dimensionality with different boundary conditions. Simpler boundary conditions would be realized by experiments with the boundary spins of relatively moving magnets aligned.
% \input{2Landauer-Buttiker}
% \input{3Full_counting_statistics}
% \input{4Heat_current}
% !TeX root = Body.tex
\chapter{Summary and Discussion}\label{chap:Summary}

We investigated the effects of boundary conditions on the physical quantities by non-equilibrium Monte Carlo simulations. To summarize the present results, we found that the fixed boundary conditions have an effect on the magnetic friction as an effective field; the anti-parallel and the parallel boundary conditions have disordering and ordering effects, respectively. These effects emerge at the sliding boundary when the system behaves as a one-dimensional system, but vanish in the two-dimensional limit. The crossover between the one dimension and the two dimensions occurs below the size $L_{z}=64$ in the limit $L_{x}\to\infty$. 

In other words, if we set the size $L_{z}$ much less (greater) than the correlation length $\xi_{z}(\beta)$ the system behaves as the one-dimensional (two-dimensional). The two sets of boundary conditions, in particular, have maximum effects on the magnetic friction when the temperature of the system is near the boundary critical temperature and the system is sufficiently thin.

Thereby we propose to manipulate the magnetic friction by switching the one boundary condition into the other near the boundary criticality. We can thus sharply increase and decrease the magnetic friction. 

In order to be more precise, we should calculate the correlation length along the $z$ direction $\xi_{z}(\beta)$. Its definition may be different from that of the equilibrium case because the homogeneity along the $z$ direction is destroyed by the constant sliding motion. A good definition of $\xi_{z}(\beta)$ could give the dimensional crossover point by the following way.

%For the temperature near the critical point, in general, the correlation length behaves as $\xi \simeq t^{-\nu}$, where $t$ and $\nu$ denote the reduced temperature $(T-T_{c})/T_{c}$ and the critical exponent, respectively. The correlation function is written by $\langle \sigma_{i}\sigma_{j}\rangle \simeq \mathrm{e}^{-|i-j|/\xi}$. This leads the following relation:
%\begin{align}
%\langle \sigma_{i}\sigma_{j}\rangle &\simeq \exp\left[-|i-j|t^{\nu}\right],\\
%\Longleftrightarrow \nu &\simeq \log\left[-\frac{\log \langle \sigma_{i}\sigma_{j}\rangle}{|i-j|}\right]/\log t\label{rel:effcritexp}.
%\end{align}
%
%The right-hand side of \eqref{rel:effcritexp} is expected to be equal to its universality class in each symmetries and dimensions. Thus we can able to discover the crossover phenomena by a value of r.h.s of \eqref{rel:effcritexp} which does not belongs to any universality class.
For further analysis, it is worthwhile to determine critical exponents of the non-equilibrium phase transition. We can determine the critical exponent $\alpha$ of the bulk heat capacity $c(T)$ in Fig.~\ref{fig:dEnDens_Allsize} as well as the boundary heat capacity $c_{\rm b}(T)$ by further calculations. Similarly we have critical exponents $\nu$, $\beta$ and $\gamma$ of the correlation length along $x$-direction $\xi_{x}$, the boundary magnetization $m_{b}$ and the boundary susceptibility $\chi_{b,\rm abs}$, respectively, by the methods in Ref.~\cite{Hucht2009b}, which enables us to discuss a crossover from the one-dimension to the two-dimension by the varying size $L_{z}$.

According to Ref.~\cite{Hucht2009b}, the scaling relation $2-\alpha=2\beta+\gamma=d_{\rm b}\nu$ where $d_{\rm b}$ denote the boundary dimension of the system for each bulk dimension and geometry. We can expect that the set of critical exponents exhibits a continuous change with holding the scaling relation.

The discussions on the experimental method of manipulating the magnetic friction and determination of the crossover point with more accuracy are applicable to any lattice system which can be simulated by our way. For more practical purposes, it is more important to study three-dimensional systems with two-dimensional surfaces. We intend to seek the way of manipulating the friction in three dimensional systems for future work.
%
%For future works, we are going to investigate the following:
%\begin{itemize}
%\item the divergence of derivatives of the \textit{boundary} heat capacity in the limit of $L_{z}\to\infty$;
%\item the behavior of the correlation length itself;
%\item the dimensional crossover from the viewpoint of critical exponents.
%\end{itemize}
%We also intend to see whether dimensional crossovers occur in models with other spatial dimensions or continuous symmetries.
\appendix
 %Appendix A %Chap3の計算詳細
% \input{App2Detailed_forchap3}
% %Appendix B
% \input{App4Sommerfeld}
%  %Appenedix C %メソ系熱機関の解析
% \input{App3heatengine}
% Appendix C

% Appendix D


% \begin{thebibliography}{99}
%
% \bibitem{goupil2011}%Thermodynamics of Thermoelectric Phenomena and Applications.
% C. Goupil, W. Seifert, K. Zabrocki, E. M\"uller, and G.~J.~ Snyder, Entropy \textbf{13}, 1481 (2011).
%
% \bibitem{callen}%Thermodynamics and an Introduction to Thermostatistics.
% H.~B.~Callen, \textit{Thermodynamics and an Introduction to Thermostatistics} (2nd Ed.), Wiley, New York (1985).
%
% \bibitem{onsager1931a}%RECIPROCAL   RELATIONS   IN   IRREVERSIBLE PROCESSES. I.
% L. Onsager, Phys. Rev. \textbf{37}, 405 (1931).
%
% \bibitem{onsager1931b}%RECIPROCAL   RELATIONS   IN   IRREVERSIBLE PROCESSES. \UTF{2161}.
% L. Onsager, Phys. Rev. \textbf{38}, 2265 (1931).
%
% \bibitem{casimir1945}%On Onsager's Principle of Microscopic Reversibility.
% H.~B.~G.~Casimir, Rev. Mod. Phys. \textbf{17}, 343 (1945).
%
% \bibitem{vandenboreck2005}%Thermodynamic Efficiency at Maximum Power.
% C. Van den Broeck, Phys. Rev. Lett. \textbf{95}, 190602 (2005).
%
% \bibitem{saito2010}%A microscopic mechanism for increasing thermoelectric efficiency.
% K.~Saito, G.~Benenti, and G.~Casati, Chem. Phys. \textbf{375}, 508 (2010).
%
% \bibitem{benenti2011}%Thermodynamic bounds on efficiency for systems with broken time-reversal symmetry.
% G.~Benenti, K.~Saito, and G.~Casati, Phys. Rev. Lett. \textbf{106}, 230602 (2011).
%
% \bibitem{saito2011}%Thermopower with broken time-reversal symmetry.
% K. Saito, G. Benenti, G. Casati, and T. Prosen, Phys. Rev. B \textbf{84}, 201306(R) (2011).
%
% \bibitem{brandner2013}%Strong Bounds on Onsager Coefficients and Efficiency for Three-Terminal Thermoelectric Transport in a Magnetic Field
% K.~Brandner, K.~Saito, and U.~Seifert, Phys. Rev. Lett. \textbf{110},  070603 (2013).
%
% \bibitem{brandner2013multi}%Multi-terminal thermoelectric transport in a magnetic field: bounds on Onsager coefficients and efficiency
% K.~Brandner and U.~Seifert, New J. Phys. \textbf{15}, 105003 (2013).
%
% \bibitem{Benenti}
% G.~Benenti, G.~Casati, T.~Prozen, and K.~Saito, arXiv:1311.4430 (2013).
%
% \bibitem{landauer1957}
% R. Landauer, IBM J. Res. \& Dev. \textbf{1}, 223 (1957).
%
% \bibitem{datta1995}%Electronic Transport in Mesoscopic Systems
% S. Datta, \textit{Electronic Transport in Mesoscopic Systems}, Cambridge University Press, Cambridge, UK (1995).
%
% \bibitem{diventra2008}%Electrical Transport in Nanoscale Systems
% M.~Di Ventra, \textit{Electrical Transport in Nanoscale Systems}, Cambridge University Press, Cambridge, UK, (2008).
%
% \bibitem{kato2014}
% T. Kato, Bussei Kenkyu Denshiban \textbf{3}, 1 (2014) (in Japanese).
%
% \bibitem{wees1988}%Quantized conductance of point contacts in a two-dimensional electron gas
% B. J. van Wees, H. van Houten, C. W. J. Beenakker, J. G. Williamson, L. P. Kouwenhoven, D. van der Marel, and C. T. Foxon, Phys. Rev. Lett. \textbf{60}, 848 (1988).
%
%  \bibitem{snyder2008}%Complex thermoelectric materials.
% G.~J.~Snyder, E.~S.~Toberer, Nat. Mater. \textbf{7}, 105 (2008).
%
% \bibitem{matthews2014}%Experimental verification of reciprocity relations in quantum thermoelectric transport.
% J. Matthews, F. Battista, D. S\'anchez, P. Samuelsson, and H. Linke, Phys. Rev. B \textbf{90}, 165428 (2014).
%
% \bibitem{izumida2012}%Efficiency at maximum power of minimally nonlinear irreversible heat engines
% Y. Izumida and K. Okuda, Europhys. Lett. \textbf{97}, 10004 (2012).
%
% \bibitem{izumida2014}%Heat devices in nonlinear irreversible thermodynamics
% Y. Izumida, K. Okuda, J.~M.~M.~Roco, and A.~C.~Hernandez, arXiv:1405.6777 (2014).
%
% \bibitem{saito2008}%Symmetry in full counting statistics, fluctuation theorem, and relations among nonlinear transport coefficients in the presence of a magnetic field
% K. Saito and Y. Utsumi, Phys. Rev. B \textbf{78}, 115429 (2008).
%
% \bibitem{nazarov2003}%Quantum Noise in Mesoscopic Physics.
% Yu.~V.~Nazarov (ed.),\textit{Quantum Noise in Mesoscopic Physics}, NATO Science Series II: Mathematics, Physics and Chemistry, Vol. \textbf{97},  Kluwer Academic, Dordrecht (2003).
%
% \bibitem{saito2009}%さいとうさんのノート
% K. Saito, Bussei Kenkyu, \textbf{92}, 345 (2009) (in Japanese).
%
% \bibitem{martin2005}%Noise in mesoscopic physics.
% T. Martin, arXiv:cond-mat/0501208 (2005).
%
% \bibitem{fetter2003}
% A.~L.~Fetter and J.~D.~Walecka, \textit{Quantum Theory of ManyParticle Systems}, Dover Publications, Inc, New York (2003).
%
% \bibitem{kurchan2001}%A Quantum Fluctuation Theorem.
% J. Kurchan, arXiv:cond-mat/0007360 (2001).
%
% \bibitem{nakamura2010}%Nonequilibrium Fluctuation Relations in a Quantum Coherent Conductor
% S. Nakamura, Y. Yamauchi, M. Hashisaka, K. Chida, K. Kobayashi, T. Ono, R. Leturcq, K. Ensslin, K. Saito, Y. Utsumi, and A.~C.~Gossard, Phys. Rev. Lett. \textbf{104}, 080602 (2010).
%
% \bibitem{nakamura2011}%Fluctuation theorem and microreversibility in a quantum coherent conductor
% S. Nakamura, Y. Yamauchi, M. Hashisaka, K. Chida, K. Kobayashi, T. Ono, R. Leturcq, K. Ensslin, K. Saito, Y. Utsumi, and A.~C.~Gossard, Phys. Rev. B \textbf{83}, 155431 (2011).
%
% \bibitem{iyoda2010}%Nonequilibrium Extension of Onsager Relations for Thermoelectric Effects in Mesoscopic Conductors.
% E. Iyoda, Y. Utsumi, and T. Kato, J. Phys. Soc. Jpn. \textbf{79}, 045003 (2010).
%
% \bibitem{ruokola2011}%Single-electron heat diode: Asymmetric heat transport between electronic reservoirs through Coulomb islands.
% T. Ruokola and T. Ojanen, Phys. Rev. B \textbf{83}, 241404(R) (2011).
%
% \bibitem{lim2013}%Dynamic thermoelectric and heat transport in mesoscopic capacitors.
% J.~S.~Lim, R.~L\'opez, and D.~S\'anchez, Phys. Rev. B \textbf{88}, 201304(R) (2013).
%
% \bibitem{sivan}
% U.~Sivan and Y. Imry, Phys. Rev. B \textbf{33}, 551 (1986).
%
% \bibitem{Butcher}
% P.~N.~Butcher, J. Phys. Condens. Matter \textbf{2}, 4869 (1990).
%
% \bibitem{jezouin2013}%Quantum limit of heat flow across a single electronic channel.
% S.~Jezouin, F.~D.~Parmentier, A.~Anthore, U.~Gennser, A.~Cavanna, Y.~Jin, and F.~Pierre, Science \textbf{342}, 601 (2013).
%
% \bibitem{iyoda}%いよださんの修論
% E.~Iyoda, \textit{Thermoelectric effects and fluctuation theorem in quantum dots}, Master Thesis, University of Tokyo (2008) (in Japanese); Bussei Kenkyu \textbf{92}, 188 (2009) (in Japanese).
%
% \bibitem{sanchez2013}%Scattering Theory of Nonlinear Thermoelectric Transport. %非線形ゼーベック
% D.~S\'anchez and R.~L\'opez, Phys. Rev. Lett. \textbf{110},  026804 (2013).
%
% \bibitem{cipiloglu2004}%Nonlinear Seebeck and Peltier effects in quantum point contacts.
% M.~A.~{\c{C}}ipilo{\u{g}}lu, S.~Turgut, and M.~Tomak, Phys. Status Solidi B \textbf{241}, 2575 (2004).
%
% \bibitem{verley2014a}%The unlikely Carnot efficiency.
% G. Verley, M. Esposito, T. Willaert, and C. Van den Broeck, Nat. Commun. \textbf{5}, 4721 (2014).
%
% \bibitem{verley2014b}%Universal theory of efficiency fluctuations.
% G. Verley, T. Willaert, C. Van den Broeck, and M. Esposito, Phys. Rev. E \textbf{90}, 052145 (2014).
%
% \bibitem{polettini2014}%Finite-time efficiency fluctuations: Enhancing the most likely value.
% M. Polettini, G. Verley, and M. Esposito, arXiv:1409.4716 (2014).
%
% \bibitem{sakurai1985}
% J.~J.~Sakurai, \textit{Modern Quantum Mechanics}, Benjamin, Menlo Park, California (1985).
%
% \bibitem{ashcroft1976}
% N.~W.~Ashcroft and N.~D.~Mermin, \textit{Solid State Physics}, Holt, Rinehart and Winston, New York (1976).
%
% \bibitem{humphrey2002}%Reversible Quantum Brownian Heat Engines for Electrons.
% T.~E.~Humphrey, R. Newbury, R.~P.~Taylor, and H. Linke, Phys. Rev. Lett. \textbf{89}, 116801 (2002).
%
% \end{thebibliography}

\end{document}
