% !TeX root = Body.tex
\chapter{Summary and Discussion}\label{chap:Summary}

In this research we obtained the result that two different fixed boundary conditions have an effect on the magnetic friction as an effective field: The anti-parallel boundary have a disordering effect and the parallel boundary have a ordering effect. In other words the anti-parallel boundary raises the temperature of the system up and the parallel boundary lowers down. These effects emerge at the sliding boundary when the system behaves as the one-dimension, but vanish when the two-dimension. In addition the crossover between the one-dimension and the two-dimension occurs below the longitudinal size $L_{z}=64$ assumed that the transverse size $L_{x}\to\infty$. Therefore we conclude that we can manipulate the magnetic friction by fixing the boundary condition depending on the temperature when the thickness of the system are enough small.

On the other hand, the efficiency of manipulation is maximized when the system is on, or at least higher than the critical point, thus it is non-trivial that the system has no structural phase transition at such the temperature and that we can consider the magnetic material as an ordinary solid state matter. Therefore, when we consider an experimental setup of the manipulation of the magnetic friction, we also have to take account for the property a the solid state matter of the magnetic material.

For future works, we are going to investigate
\begin{itemize}
\item Consistency between the case of free boundary condition to the results by Hucht\cite{Hucht2009b}
\item Divergence of derivatives of the \textit{boundary} energy (or boundary capacity) for the limit of two-dimension
\item Behavior of the correlation length itself
\item Dimensional crossover of the critical exponents of $\partial f(L_{z},T)/\partial T$ and $\partial \epsilon_{\rm b}/\partial T$
\end{itemize}
to analyze the phenomena in more detail. We intend to see whether dimensional crossovers in models with other spatial dimensions or continuous symmetries.