% !TeX root = Body.tex
\chapter{Summary and Discussion}\label{chap:Summary}

To summarize the present results, we found that two different fixed boundary conditions have an effect on the magnetic friction as an effective field; the anti-parallel and the parallel boundary conditions have disordering and ordering effects, respectively. In other words, the anti-parallel boundary conditions raise the effective temperature of the system and the parallel boundary conditions lower it. These effects emerge at the sliding boundary when the system behaves as a one-dimensional system, but vanish in the two-dimensional limit. The crossover between the one dimension and the two dimensions occurs below the size $L_{z}=64$ in the limit $L_{x}\to\infty$. 

The difference of the magnetic frictions under the two boundary conditions is maximized when the temperature of the system is near the boundary critical temperature and the system is sufficiently thin. Therefore we propose to manipulate the magnetic friction by switching the one boundary condition into the other of the system in the boundary criticality.

For future works, we are going to investigate the following:
\begin{itemize}
\item the divergence of derivatives of the \textit{boundary} heat capacity in the limit of $L_{z}\to\infty$;
\item the behavior of the correlation length itself;
\item the dimensional crossover from the viewpoint of critical exponents.
\end{itemize}
We also intend to see whether dimensional crossovers occur in models with other spatial dimensions or continuous symmetries.