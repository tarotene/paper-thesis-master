% !TeX root = Body.tex
\chapter{Summary and Discussion}\label{chap:Summary}

We investigated the effects of boundary conditions on the physical quantities by non-equilibrium Monte Carlo simulations. To summarize the present results, we found that the fixed boundary conditions have an effect on the magnetic friction as an effective field; the anti-parallel and the parallel boundary conditions have disordering and ordering effects, respectively. These effects emerge at the sliding boundary when the system behaves as a one-dimensional system, but vanish in the two-dimensional limit. The crossover between the one dimension and the two dimensions occurs below the size $L_{z}=64$ in the limit $L_{x}\to\infty$. 

In other words, if we set the size $L_{z}$ much less (greater) than the correlation length $\xi_{z}(\beta)$ the system behaves as the one-dimensional (two-dimensional one). The two sets of boundary conditions, in particular, have maximum effects on the magnetic friction when the temperature of the system is near the boundary critical temperature and the system is sufficiently thin.

Thereby we propose to manipulate the magnetic friction by switching the one boundary conditions into the other near the boundary criticality. We can thus sharply increase and decrease the magnetic friction. 

In order to be more precise, we should calculate the correlation length along the $z$ direction $\xi_{z}(\beta)$. Its definition may be different from that of the equilibrium case because the homogeneity along the $z$ direction is destroyed by the constant sliding motion.

%For the temperature near the critical point, in general, the correlation length behaves as $\xi \simeq t^{-\nu}$, where $t$ and $\nu$ denote the reduced temperature $(T-T_{c})/T_{c}$ and the critical exponent, respectively. The correlation function is written by $\langle \sigma_{i}\sigma_{j}\rangle \simeq \mathrm{e}^{-|i-j|/\xi}$. This leads the following relation:
%\begin{align}
%\langle \sigma_{i}\sigma_{j}\rangle &\simeq \exp\left[-|i-j|t^{\nu}\right],\\
%\Longleftrightarrow \nu &\simeq \log\left[-\frac{\log \langle \sigma_{i}\sigma_{j}\rangle}{|i-j|}\right]/\log t\label{rel:effcritexp}.
%\end{align}
%
%The right-hand side of \eqref{rel:effcritexp} is expected to be equal to its universality class in each symmetries and dimensions. Thus we can able to discover the crossover phenomena by a value of r.h.s of \eqref{rel:effcritexp} which does not belongs to any universality class.
For further analysis, it is worthwhile to determine critical exponents of the non-equilibrium phase transition. We can determine the critical exponent $\alpha$ of the bulk heat capacity $c(T)$ in Fig.~\ref{fig:dEnDens_Allsize} as well as the boundary heat capacity $c_{\rm b}(T)$ by further calculations. Similarly we can estimate the critical exponents $\nu$, $\beta$ and $\gamma$ of the correlation length $\xi_{x}$ along $x$ direction, the boundary magnetization $m_{\rm b}$ and the boundary susceptibility $\chi_{\rm b, abs}$, respectively, by the methods in Ref.~\cite{Hucht2009b}, which enables us to discuss a crossover from the one dimension to two dimensions by varying the size $L_{z}$.

According to Ref.~\cite{Hucht2009b}, the scaling relation $2-\alpha=2\beta+\gamma=d_{\rm b}\nu$ exists, where $d_{\rm b}$ denote the boundary dimension of the system for each bulk dimension and geometry. We can expect that the set of critical exponents exhibits a continuous change with the scaling relation holding.

The discussions on the experimental method of manipulating the magnetic friction and determination of the crossover point with more accuracy are applicable to any lattice system which can be simulated by our way. For more practical purposes, it is important to study three-dimensional systems with two-dimensional surfaces. We intend to seek the way of manipulating the friction in three-dimensional systems for future work.
%
%For future works, we are going to investigate the following:
%\begin{itemize}
%\item the divergence of derivatives of the \textit{boundary} heat capacity in the limit of $L_{z}\to\infty$;
%\item the behavior of the correlation length itself;
%\item the dimensional crossover from the viewpoint of critical exponents.
%\end{itemize}
%We also intend to see whether dimensional crossovers occur in models with other spatial dimensions or continuous symmetries.