% !TeX root = Body.tex
\chapter{Introduction}
%The system is one of the simplest model of two-dimensional magnetic friction. Its spatial and spin dimensionality are far from realistic materials around us. However, we can use several facts from the exact solution for the two-dimensional Ising model, which makes the analysis easier than higher-dimensional cases.

%In this chapter we introduce two of the most famous problems in analyzing the frictional force as problems of statistical mechanics, giving recent developments for solving them. We then pose another problem regarding \textit{manipulation} of the sliding friction which occurs in highly lubricated solids. To this end we simplify our problem into a dimensional crossover in lattice systems.
%
%The sliding friction in solids is a very complicated problem to analyze, despite the fact that our daily lives are linked with it in various forms. One reason of the difficulty is that there is no general theory which determines the most important microscopic degree of freedom to describe the macroscopic phenomenon of the sliding friction.
%
%\section{Sliding Frictions as Non-Equilibrium Problems}\label{sec:NESliding}
%
%
%
%Several model calculations and experiments have solved these problems, directly \cite{Weiss1996,Weiss1997,Garcia-Mata2007,Meyer2015,Bylinskii2016,Novotny1999,Hosomi2007,Hosomi2008,Hosomi2009,Ternes2008,Urbakh2010,Kim2007,Grierson2007,Manini2017,Kadau2008,Magiera2011,Wolter2012,Hucht2009b} and indirectly \cite{Strunz1998a,Strunz1998,Novaco2015b,Meng2015}. Phononic contributions to sliding frictions were analyzed with the Frenkel-Kontrova model \cite{Weiss1996,Weiss1997,Strunz1998a,Strunz1998,Novaco2015b,Meng2015} and simulated with cold atoms \cite{Garcia-Mata2007,Meyer2015,Bylinskii2016}, whereas eletronic contributions were discovered \cite{Dayo1998} and explained within an image potential theory \cite{Novotny1999}. As a quantum phenomena, sliding friction between adsorbed helium atoms and its substrates \cite{Hosomi2007,Hosomi2008,Hosomi2009} is recently getting attention, but their theoretical description is still not satisfactory and remains to be an open problem.
%
%\section{Impossibility of the Observation of the Sliding Surface}\label{sec:IMPObsSlSf}
%The dimensionality of the sliding surface is up to two, if that of the whole system is three. Sliding surfaces are different in many ways from the standard two-dimensional surfaces on three-dimensional solids which have been investigated for many years. In particular we cannot perform a direct observation of the sliding surface by apparatuses such as the standard microscope. The difficulty has prevented us from clarifying non-equilibrium properties of the sliding friction.
%
%The frictional-force microscope have played important roles in investigating such sliding surfaces. This apparatus sees the roughness of the substrate by observing frictional force on its tips, by which we can recover the macroscopic frictional force as its integration over the surface. By the use of frictional-force microscopes, a local lateral force was measured by Ternes \textit{et al}.\ \cite{Ternes2008} and \textit{superlubricity} was observed by Dienwiebel \cite{Dienwiebel2004} as an intriguing phenomenon in which the frictional force is strongly dependent on the sliding direction and almost vanishes in several directions.
%
%\section{Manipulating the Friction}
%
%As mentioned above, recent researches well revealed the nature of sliding frictions. This gave rise to new problems about the friction in atomically microscopic systems.
%
%Ordinary frictions in solids are mostly governed by excitations of phonon degrees of freedom, because the sliding surface is almost always rougher than the scale of the atom. Once we lubricate the sliding surface highly, other degrees of freedom such as the orbital and the spin angular momenta of electrons emerge as additional contributions to the friction.
%
%We are already familiar with the most remarkable example of such a system in our daily lives, namely \textit{micro-electric mechanical systems} (\textbf{MEMS}). MEMS plays important roles in the printing head of inkjet printers, the accelerometer in smartphones and so on. MEMS has clean planes sliding with an accuracy of a micrometer or a nanometer \cite{Kim2007,Grierson2007,Manini2017}, because of which they experience the friction with many kinds of degrees of freedom including not only phonon excitations but also other contributions such as  excitaions and electron-phonon couplings. The small size of these systems results in a large friction, which lessens the working efficiency or disables the operation of MEMS, because the ratio of the surface area over the volume of the system becomes larger with the smaller size under a fixed geometry. In order to tackle with the issue of manipulation of the friction in small systems, we need to obtain more fundamental knowledge of the friction.
%
%\section{Magnetic Friction}
%The way to manipulate the friction in small systems is less understood than its nature. We here consider manipulating friction of magnetic materials by numerically simulating lattice models. 
%
%The nature of magnetic frictions was first revealed by Kadau \textit{et al}. \cite{Kadau2008}\ by the use of the Ising model. Just after it, Fusco \textit{et al}. \cite{Fusco2008} investigated magnetic frictions in the Heisenberg spin systems and a moving dipole moment on them. A crossover of the velocity dependence of the magnetic frictional force was revealed by Magiera \textit{et al}. \cite{Magiera2011}. A direct observation of the magnetic frictions by a scanning probe microscopy was then performed by Wolter \textit{et al}. \cite{Wolter2012}.
%
%In the present work we consider two strips of the quasi-one-dimensional Ising model sliding against each other with a fixed velocity. We discuss the difference of the frictional forces under two boundary conditions, namely the anti-parallel and the parallel, and its dependence on the distance  between the two boundaries. We find a method of manipulating the friction using the  different boundary conditions. These boundary conditions can be realized in experiments  by aligning the boundary spins of sliding magnets.
%
%Frictions in such models were considered first in a numerical study by Kadau \textit{et al}. \cite{Kadau2008}. They revealed by Monte Carlo simulations that two square lattices of the Ising model which slide against each other experience the friction, depending on the temperature and the sliding velocity. Immediately after the research, it was revealed \cite{Hucht2009b} that the Ising model goes under a non-trivial non-equilibrium phase transition (\textbf{NEPT}) in the high-velocity limit, where the two square lattices are decoupled in terms of the correlation between the two lattices and feel an effective mean field depending on the magnetization of each other \cite{Hucht2009b}. This analysis showed that a novel critical point exists at a generally higher temperature than the ordinal critical point for models in arbitrary dimensions and geometries (see Chapter~\ref{ch:review}). They also developed a new dynamics \cite{Hucht2009b} which enables an analytical treatment for finite sliding velocities, obtaining a non-equilibrium critical line.
%

From old days, frictional forces between two solid-state matters have been thought to obey the following three fundamental laws---the Amontons-Coulomb law \cite{Matsukawa2009Book}:
\begin{enumerate}
	\item The frictional force is independent of the size of the \textit{apparent} contact area;
	\item The frictional force is proportional to the normal force on the sliding body;
	\item The dynamic frictional force is independent of the sliding velocity and is less than the static one.
\end{enumerate}
The former two laws hold irrespective of static or dynamic force. These two laws imply that the frictional coefficient, defined by the ratio of the frictional force and the load of the sliding body, is a constant.

However, it began to be known that several experimental facts violate the Amontons-Coulomb law, particularly the third law about the velocity independence, depending on system parameters such as the temperature \cite{Bouhacina1997,Gnecco2000,Chen2006,Muser2011,Braun2011}. The velocity dependence of the friction shows, in general, quite different aspects for regions of very large velocities and very small ones.

A new wider framework of frictional forces, called the \textit{constitutive equation of friction} \cite{Hashiguchi2016}, is thought to cover the frictional phenomena irrespectively of whether the Amontons-Coulomb law holds or not. The constitutive equation of friction improves the third law, describing the velocity dependence of the frictional force. This framework assumes that the frictional coefficient obeys the following form:
\begin{align}
\mu = \mu_{0} + A\log \left[1+\frac{v}{v_{0}}\right] + B\log \left[1+\frac{\theta}{\theta_{0}}\right],
\end{align}
where $A$, $B$, $v_{0}$ and $\theta_{0}$ are constants, while $v$ and $\theta$ denote the sliding velocity and a \textit{state variable}, respectively. The constitutive equation of motion, as obvious from its form, describes the logarithmic dependence of the frictional coefficient on the velocity. We generally impose the constraint $d\theta/dt = 1-\theta v/D_{\rm c}$ on the variables $v$ and $\theta$, where $D_{\rm c}$ is a constant. If we make the sliding object have a constant velocity $v$ the time dependences of $v$ and $\theta$ vanish and the relation $D_{\rm c}=\theta v$ always holds, and hence the coefficient behaves as $\mu={\rm const.} +(A-B)\log v$ in the regime sufficiently large velocity. This framework includes other parameters such as the temperature in a natural manner.

From the viewpoint of the interaction, we understand the velocity dependence of the friction in the following way. We consider an object $O$ and a substrate $S$, letting $O$ slide against $S$ by applying an external force $F_{\rm ext}$ to $O$. When $O$ and $S$ interact with each other, the kinetic energy of $O$ given by the external force is expected to be lost through the interaction, and then the entire system of $O$ and $S$ heats up (if the system is closed) or the energy dissipates from the system to an external environment (if the system is open). In the latter case, when we control the external force $F_{\rm ext}$ to balance it with the frictional force $F_{\rm fric}$ the sliding velocity $v$ becomes constant and the dissipation process becomes stationary. Then the frictional force $F_{\rm fric}$ can be considered as a function of the sliding velocity $v$. If we set the temperature of the environment constant, we can also discuss the temperature dependence in the stationary regime of an open system.

It is also important to understand macroscopic frictional phenomena by fundamental laws of microscopic physical degrees of freedom.
Nonetheless, many conventional ways of statistical mechanics are not applicable directly to the frictional phenomena. For example, the linear-response theory appears to solve the problem if the velocity $v$ is much smaller than the rate $\xi/\tau$, where $\xi$ and $\tau$ are the characteristic length and time of the system, respectively. However, we already know well that the static frictional force is non-zero for many systems, for which the frictional force has a non-linear velocity dependence. This shows us that the complexity of the problem is beyond the linear-response theory.

For this reason, many theoretical researches on frictional phenomena have been performed by means of numerical simulations and approximately solvable models. For example, sliding frictions with a logarithmic dependence on the velocity have been investigated in terms of the Prandtl-Tomlinson model, which describes a total stick-slip motion as an elementary process \cite{Li2011,Sang2001,Schirmeisen2005,Muser2011,Xu2011,Jansen2010}. As a many-body description of the stick-slip motion, researches using the Frenkel-Kontrova model also has been performed \cite{Braun2011,Elmer1994a,Zheng1998,Bhattacharya2013}. A novel frictional phenomenon of vanishing frictional force, namely \textit{superlubricity}, is well understood by the Frenkel-Kontrova model \cite{Wang2008,Benassi2015}. 

Frictional phenomena are not captured only by classical picture, such as a moving mass interacts with a sinusoidal potential. In recent researches, the contribution of conduction electrons in superconductors to the friction (\textbf{electronic friction}) found to drop abruptly near the superconducting $T_{\rm c}$ \cite{Dayo1998,Novotny1999,Qi2008} and that of localized spins in magnetic materials (\textbf{magnetic friction}) was investigated by numerical methods for macroscopic systems \cite{Kadau2008,Fusco2008,Magiera2011} and observed by the scanning tunneling microscopy \cite{Wolter2012}.

% electron motions and its spin degrees of freedom on the frictional force are investigated both experimentally and theoretically. Some of their properties has been revealed by numerical calculations with the assumption that the contact area is completely lubricated and there is no lattice vibration. Their frictional phenomena belong to the category which violates the first and third laws of the Amontons and Coulomb's law.

In the present work we focus on the spin degrees of freedom and introduce a new viewpoint of manipulating the friction by an external operation. General ways of manipulating the friction in the scales of atoms and molecules have been already discussed. However, almost all of them are limited to mechanical ways, such as reducing the actual load to realize a smaller normal force and using intrinsic properties of the system such as the superlubricity. In contrast, we use the boundary conditions as an effective field which couples directly with the spin degrees of freedom in order to control the frictional force. Our way of manipulating the friction is applicable not only to spin systems but widely to electron systems and so on, opening up a new way of manipulating the friction.

%In the present work, we focus on the following two results. Kadau et al.\ [] first confirmed the Ising spin systems have sliding frictions as a result of repeated destruction and reconnection of the interactions between two spins at a sliding boundary by Monte Carlo simulations. The frictional force i.) takes the highest value at a temperature near the critical temperature, and decays ii.) to exponentially and iii.) to a power at the zero temperature and the infinity temperature, respectively. The result ii. correspond to the fact that the full magnetized ground state emerges at the zero temperature as the equilibrium state and the sliding operation to the Ising model makes no energy changes. The result iii. describes the full randomness of the system at the infinite temperature, which makes the sliding operation still meaningless. In addition, the boundary magnetization at the sliding surface was found to become larger than that in the bulk part for a range of the temperature. This sliding-induced boundary magnetizations were exactly investigated in the following year by Hucht [] (see the review of Ref.~[] in Chapter~\ref{}).

The setup is in the following way. We consider two strips of the quasi-one-dimensional Ising model sliding against each other with a fixed velocity. Note that we assume the perfectly contacting two bodies, which produces the frictional force proportional to the contact area, and consider the dynamical friction (hereinafter simply called ``friction") in a stationary regime far from equilibrium. We discuss the difference of the frictional forces under two boundary conditions, namely the anti-parallel and the parallel, and its dependence on the distance between the two boundaries. We show a relation between the manipulativeness of the magnetic friction  and the dimensionality of the system. In the one-dimensional limit the boundary conditions seem to have the maximum effect on the friction, whereas in the two-dimensional limit there seems to be no effects.

This paper is organized as follows. In a brief review in Chapter~\ref{ch:review}, we provides the fact that Ising models with any spatial dimension and geometry have a non-equilibrium phase transition, which occurs generally at a higher temperature than its equilibrium case. Chapter~\ref{chap:NumSim} provides the setup of our model in detail and definitions of physical quantities in both the non-equilibrium regime and the equilibrium one. In Chapter~\ref{chap:Res}, we discuss the results from our simulation in association with the non-equilibrium phase transitions in Chapter~\ref{ch:review}. Our conclusions are summarized in Chapter~\ref{chap:Summary}.
 
%We find a method of manipulating the friction using the different boundary conditions. These boundary conditions can be realized in experiments by aligning the boundary spins of sliding magnets.

