% !TeX root = Body.tex
\chapter{Introduction}
%The system is one of the simplest model of two-dimensional magnetic friction. Its spatial and spin dimensionality are far from realistic materials around us. However, we can use several facts from the exact solution for the two-dimensional Ising model, which makes the analysis easier than higher-dimensional cases.
In this chapter we introduce two of the most famous problems in analyzing the frictional force as problems of statistical mechanics, giving recent developments for solving them. We then pose another problem regarding \textit{manipulation} of the sliding friction which occurs in highly lubricated solids. To this end we simplify our problem into a dimensional crossover in lattice systems.

The sliding friction in solids is a very complicated problem to analyze, despite the fact that our daily lives are linked with it in various forms. One reason of the difficulty is that there is no general theory which determines the most important microscopic degree of freedom to describe the macroscopic phenomenon of the sliding friction.

One may think that with the skill of statistical mechanics we can deal with the problem in a systematic manner. However, there are several essential problems including the following: (i) The sliding friction is essentially a non-equilibrium phenomenon (see Section~\ref{sec:NESliding}); (ii) We cannot directly observe the sliding surface (see Section~\ref{sec:IMPObsSlSf}).

\section{Sliding Frictions as Non-Equilibrium Problems}\label{sec:NESliding}
We can regard the sliding friction as the following elementary problem. We consider an object $O$ and a substrate $S$, and let $S$ slide against $O$ by applying an external force $f_{\rm ext}$ to $O$. When $O$ and $S$ interact with each other, the kinetic energy of $O$ given by the external force is expected to be lost through the interaction, and then the entire system of $O$ and $S$ heats up (if the system is closed) or the energy dissipates from the system to an external environment (if the system is open). In the latter case, when we control the external force $f_{\rm ext}$ to balance it with the frictional force $f_{\rm fric}$ the sliding velocity $v$ becomes constant and the dissipation process becomes stationary. Then the frictional force $f_{\rm fric}$ can be considered as a function of the sliding velocity.

The standard method of statistical mechanics called the linear-response theory appears to solve the problem if the velocity is much smaller than the rate $\xi/\tau$, where $\xi$ and $\tau$ are the characteristic length and time of the system, respectively. However, we already know well that the static frictional force is non-zero for several systems, for which the frictional force has a non-linear velocity dependence. This shows us that the complexity of the problem is beyond the linear-response theory.

Several model calculations and experiments have solved these problems, directly \cite{Weiss1996,Weiss1997,Garcia-Mata2007,Meyer2015,Bylinskii2016,Novotny1999,Hosomi2007,Hosomi2008,Hosomi2009,Ternes2008,Urbakh2010,Kim2007,Grierson2007,Manini2017,Kadau2008,Magiera2011,Wolter2012,Hucht2009b} and indirectly \cite{Strunz1998a,Strunz1998,Novaco2015b,Meng2015}. Phononic contributions to sliding frictions were analyzed with the Frenkel-Kontorova model \cite{Weiss1996,Weiss1997,Strunz1998a,Strunz1998,Novaco2015b,Meng2015} and simulated with cold atoms \cite{Garcia-Mata2007,Meyer2015,Bylinskii2016}, whereas eletronic contributions were discovered \cite{Dayo1998} and explained within an image potential theory \cite{Novotny1999}. As a quantum phenomena, sliding friction between adsorbed helium atoms and its substrates \cite{Hosomi2007,Hosomi2008,Hosomi2009} is recently getting attention, but their theoretical description is still not satisfactory and remains to be an open problem.

\section{Impossibility of the Observation of the Sliding Surface}\label{sec:IMPObsSlSf}
The dimensionality of the sliding surface is up to two, if that of the whole system is three. Sliding surfaces are different in many ways from the standard two-dimensional surfaces on three-dimensional solids which have been investigated for many years. In particular we cannot perform a direct observation of the sliding surface by apparatuses such as the standard microscope. The difficulty has prevented us from clarifying non-equilibrium properties of the sliding friction.

The frictional-force microscope have played important roles in investigating such sliding surfaces. This apparatus sees the roughness of the substrate by observing frictional force on its tips, by which we can recover the macroscopic frictional force as its integration over the surface. By the use of frictional-force microscopes, a local lateral force was measured by Ternes \textit{et al}.\ \cite{Ternes2008} and \textit{superlubricity} was observed by Urbakh \cite{Urbakh2010} as an intriguing phenomenon in which the frictional force is strongly dependent on the sliding direction and almost vanishes in several directions.

\section{Manipulating the Friction}

As mentioned above, recent researches well revealed the nature of sliding frictions. This gave rise to new problems about the friction in atomically microscopic systems.

Ordinary frictions in solids are mostly governed by excitations of phonon degrees of freedom, because the sliding surface is almost always rougher than the scale of the atom. Once we lubricate the sliding surface highly, other degrees of freedom such as the orbital and the spin angular momenta of electrons emerge as additional contributions to the friction.

We are already familiar with the most remarkable example of such a system in our daily lives, namely \textit{micro-electric mechanical systems} (\textbf{MEMS}). MEMS plays important roles in the printing head of inkjet printers, the accelerometer in smartphones and so on. MEMS has clean planes sliding with an accuracy of a micrometer or a nanometer \cite{Kim2007,Grierson2007,Manini2017}, because of which they experience the friction with many kinds of degrees of freedom including not only phonon excitations but also other contributions such as electronic excitaions and electron-phonon couplings. The small size of these systems results in a large friction, which lessens the working efficiency or disables the operation of MEMS, because the ratio of the surface area over the volume of the system becomes larger with the smaller size under a fixed geometry. In order to tackle with the issue of manipulation of the friction in small systems, we need to obtain more fundamental knowledge of the friction.

\section{Magnetic Friction}
The way to manipulate the friction in small systems is less understood than its nature. We here consider manipulating friction of magnetic materials by numerically simulating lattice models. 

The nature of magnetic frictions was first revealed by Kadau \textit{et al}. \cite{Kadau2008}\ by the use of the Ising model. Just after it, Fusco \textit{et al}. \cite{Fusco2008} investigated magnetic frictions in the Heisenberg spin systems and a moving dipole moment on them. A crossover of the velocity dependence of the magnetic frictional force was revealed by Magiera \textit{et al}. \cite{Magiera2011}. A direct observation of the magnetic frictions by a scanning probe microscopy was then performed by Wolter \textit{et al}. \cite{Wolter2012}.

In the present work we consider two strips of the quasi-one-dimensional Ising model sliding against each other with a fixed velocity. We discuss the difference of the frictional forces under two boundary conditions, namely the anti-parallel and the parallel, and its dependence on the distance  between the two boundaries. We find a method of manipulating the friction using the  different boundary conditions. These boundary conditions can be realized in experiments  by aligning the boundary spins of sliding magnets.

Frictions in such models were considered first in a numerical study by Kadau \textit{et al}. \cite{Kadau2008}. They revealed by Monte Carlo simulations that two square lattices of the Ising model which slide against each other experience the friction, depending on the temperature and the sliding velocity. Immediately after the research, it was revealed \cite{Hucht2009b} that the Ising model goes under a non-trivial non-equilibrium phase transition (\textbf{NEPT}) in the high-velocity limit, where the two square lattices are decoupled in terms of the correlation between the two lattices and feel an effective mean field depending on the magnetization of each other \cite{Hucht2009b}. This analysis showed that a novel critical point exists at a generally higher temperature than the ordinal critical point for models in arbitrary dimensions and geometries (see Chapter~\ref{ch:review}). They also developed a new dynamics \cite{Hucht2009b} which enables an analytical treatment for finite sliding velocities, obtaining a non-equilibrium critical line.

Based on the results in Ref.~\cite{Hucht2009b}, we consider a dimensional crossover from one dimension to two in Ising models with the two boundary conditions. In the one-dimensional limit the boundary conditions seem to have the maximum effect on the friction, whereas in the two-dimensional limit there seems to be no effects. The behaviors in the both limits for the free boundary conditions correspond to the results in Ref.~\cite{Hucht2009b}. 