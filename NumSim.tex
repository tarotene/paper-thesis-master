% !TeX root = Body.tex
\chapter{Numerical Simulations}\label{chap:NumSim}

\section{Setup of the Model}\label{sec:SetupModel}
Sliding friction is a form of energy dissipation on the surface between a moving object and its substrate. The dissipated energy is originated in the kinetic energy of the moving object. We here consider a constantly moving case in which an external force maintains the motion of the object with endless supply of its kinetic energy. This view leads to its \textit{non-equilibrium stationary state}. When the system is in a non-equilibrium stationary state, it is often easy to calculate \textit{energy currents} such as the frictional heat, its power and so on. Applying the view to our case in which two square lattices of the Ising model slide against each other, we can formulate the problem as follows; see Fig.~\ref{fig:CutIsing}.
\begin{enumerate}
	\item We prepare a square lattice of the Ising model of size $L_{x}\times L_{z}$ and impose periodic boundary conditions in the transverse ($x$) direction. We first set the system in the equilibrium state of a temperature $T$, whereas we set the open boundary conditions in the longitudinal ($z$) direction for the moment.
	\item We cut the system along the $x$-direction into two parts, maintaining interactions on the cut.
	\item We slide two parts along the cut plane with relative velocity $v$. In other words, we shift the upper half by a lattice constant every $1/v$ unit time. 
\end{enumerate}

\begin{figure}[htbp]
	\centering
	\includegraphics[width=0.25\linewidth]{../../Slides/Ingredients/03-CutIsing.pdf}
	\caption{Two cylinders of the Ising model sliding with the velocity $v$.}
	\label{fig:CutIsing}
\end{figure}

The Hamiltonian of the system is given by
\begin{align}
&\hat{H}=\hat{H}_{\rm upper} + \hat{H}_{\rm lower} + \hat{H}_{\rm slip}(t),
\end{align}
where
\begin{align}
&\hat{H}_{\rm upper}:=-J\sum_{\langle i,j\rangle\in\mathrm{upper}}\hat{\sigma}_{i}\hat{\sigma}_{j}\label{ham:upper}, \\
&\hat{H}_{\rm lower}:=-J\sum_{\langle i,j\rangle\in\mathrm{lower}}\hat{\sigma}_{i}\hat{\sigma}_{j}\label{ham:lower}, \\
&\hat{H}_{\rm slip}(t):=-J\sum_{\langle i,j(t)\rangle\in\mathrm{slip}}\hat{\sigma}_{i}\hat{\sigma}_{j(t)}\label{ham:slip}
\end{align}
upper, lower and slip representing the set of interacting spin pairs on the upper half, the lower half and the slip plane, respectively.
Shift operations lead the system to repeated \textit{pumping} and \textit{dissipation} processes as follows:
\begin{enumerate}
	\item \textbf{Shift}: A shift operation excites the energy on the slip plane by the amount $\langle\hat{H}_{\rm slip}(t')-\hat{H}_{\rm slip}(t)\rangle_{\rm st}$. The letter $t'$ denotes the time just after the shift operation at time $t$.
	\item \textbf{Relax-1}: The excited energy on the slip plane $\langle\hat{H}_{\rm slip}(t')-\hat{H}_{\rm slip}(t)\rangle_{\rm st}$ dissipates to the entire system.
	\item \textbf{Relax-2}: The excited entire system relaxes towards the equilibrium.
\end{enumerate}
We defined the stationary state average $\langle\hat{A}\rangle_{\rm st}:=\sum_{i}A_{i}p^{(\rm st)}_{i}$ for an arbitrary observable $\hat{A}$, where $\{A\}_{i}$ are eigenvalues of $\hat{A}$ and $\{p^{(\rm st)}_{i}\}$ is the stationary-state probability distribution, which is different from the equilibrium (canonical) probability distribution $p^{(\rm eq)}_{i}\propto\exp\left[-E_{i}/k_{\rm B} T\right]$. Note that the distribution $\{p^{(\rm st)}_{i}\}$ depends on the sliding velocity $v$. 

The excited and relaxed amounts of energy per unit time correspond to the energy pumping and dissipation, respectively. The energy pumping $P(t)$ and dissipation $D(t)$ are given by
\begin{align}
P(t):=&\sum_{i_{v}=0}^{v-1}\left\langle \hat{H}_{\rm slip}\left(t'-1+\frac{i_{v}}{v}\right) - \hat{H}_{\rm slip}\left(t-1+\frac{i_{v}}{v}\right)\right\rangle_{\rm st},\\
D(t):=&\sum_{i_{v}=0}^{v-1}\left\langle \hat{H}_{\rm slip}\left(t-1+\frac{i_{v}+1}{v}\right) - \hat{H}_{\rm slip}\left(t'-1+\frac{i_{v}}{v}\right)\right\rangle_{\rm st},
\end{align}
respectively. $P(t)$ and $D(t)$ correspond to the energy difference due to the \textbf{Shift} and the \textbf{Relax} processed, respectively. Note that absolute values of $P(t)$ and $D(t)$ become equal to each other in the non-equilibrium stationary state, by its definition.

\section{Definitions of Physical Quantities}
We now consider the case in which the system is in a non-equilibrium stationary state. We denote by $P(L_{x}, L_{z}, T)$ and $D(L_{x}, L_{z}, T)$ the long-time limit of energy pumping $P(t)$ and dissipation $D(t)$ for a system of size $L_{x}\times L_{z}$ at the temperature $T$. We define the frictional force density $f(L_{z}, T)$ by
\begin{align}
f(L_{z}, T):=\lim_{L_{x}\to\infty}\frac{F(L_{x}, L_{z}, T)}{L_{x}}.
\end{align}

In numerical simulations, we calculate the frictional force $F(L_{x}, L_{z}, T)$ using its power $D(L_{x}, L_{z}, T)$ by the formula
\begin{align}
F(L_{x}, L_{z}, T)=\frac{D(L_{x}, L_{z}, T)}{v}\label{for:frictionalforce}.
\end{align}

We can easily verify the formula \eqref{for:frictionalforce} by considering general cases in which the frictional force and its power are both time dependent. Denoting the frictional force $F(x)$ at the position $x$, it holds that
\begin{align}
\int_{t_{0}}^{t_{1}}dt\;D(t)=\int_{x(t_{0})}^{x(t_{1})}dx\;F(x)=\int_{t_{0}}^{t_{1}}\frac{dx}{dt}dt\;F(x(t))=v\int_{t_{0}}^{t_{1}}dt\;F(x(t))\label{rel:PowerFrictionalforce}
\end{align}
for a time dependent $D(t)$, because $dx/dt=v$. Under the assumption of a non-equilibrium stationary state, the integrands in both-hand sides of the relation \eqref{rel:PowerFrictionalforce} are still equal to each other in the long-time limit, and hence 
\begin{align}
D(L_{x}, L_{z}, T)=vF(L_{x}, L_{z}, T).
\end{align}
From now we call the quantity $D(L_{x}, L_{z}, T)$ the \textit{energy dissipation}.

Our model always reaches a non-equilibrium stationary states in the long-time limit $t\to\infty$, which depends on the temperature $T$ and the sliding velocity $v$; We will prove it in .~\ref{chap:ProofEx}. We use the fact that $\lim_{t\to\infty}|D(t)|=\lim_{t\to\infty}|P(t)|$ in order to estimate the average $\bar{D}$; the average $\bar{P}$ has less statistical fluctuation~\cite{Magiera2009a, Magiera2011, Magiera2011b}. We therefore have
\begin{align}
P(L_{x}, L_{z}, T)=vF(L_{x}, L_{z}, T)\label{for:frictionalforce2}.
\end{align}
We also define the bulk energy density $\epsilon_{\rm b}(L_{z}, T)$ as follows:
\begin{align}
\epsilon_{\rm b}(L_{z}, T):=\lim_{L_{x}\to\infty}\frac{E_{\rm b}(L_{x}, L_{z}, T)}{L_{x}L_{z}},
\end{align}
where $E_{\rm b}(L_{x}, L_{z}, T)$ is the energy of the entire system. From this, we define the bulk heat capacity $c_{\rm b}(L_{z}, T)$ as follows:
\begin{align}
c_{\rm b}(L_{z}, T):=\frac{\partial \epsilon_{\rm b}(L_{z}, T)}{\partial T}.
\end{align}

\section{Non-equilibrium Monte Carlo Simulation}
The dissipation process towards the heat bath occurs via a spin flip. This fundamental processes do not only describe equilibrium states but also non-equilibrium stationary states at a fixed temperature $T$~\cite{Glauber1963}. Using Monte Carlo method, we simulate this process.

\subsection{Introduction the Time Scale to Ising Models}
In order to calculate dynamical observables such as the frictional power \eqref{for:frictionalforce2} and its dissipation rate \eqref{for:frictionalforce}, we have to define \textit{a unit time} for finite size systems. 

For the equilibrium Monte Carlo simulation, the most naive approach for the equilibrium state is the single-spin-flip algorithm, where we perform the sequence of a random selection of a spin and its flip with a temperature dependent probability $p(T)$. Whatever we use as the probability for the Monte Caro simulation should satisfy a good property, called \textit{detailed balanced condition}, which certainly leads the system towards the true equilibrium state with enough repetition of the sequence. For example, we often use the Metropolis probability $p_{\rm M}(T):=\min\{1,\mathrm{e}^{-\frac{\Delta E}{k_{\rm B}T}}\}$ as the probability $p(T)$, where $\Delta E$ is the energy difference due to the flip. We often call a \textit{Monte Carlo step} a single process of the algorithm, and define a \textit{Monte Carlo sweep} by $N$ Monte Carlo steps, where $N$ is the number of spins.

Which do we have to define a unit time by a Monte Carlo step or a Monte Carlo sweep? Its answer can be seen in the following manner: We often assume that a statistical mechanical model are coupling to a heat bath by each local degree of freedom. The temperature of the system is kept constant by the heat bath and exchanges its energy with the heat bath through local degrees of freedom. It is most natural to assume that the frequency of the exchanging process is dependent only on the temperature of the heat bath. Thus the total number of times exchanged are proportional to the number of degrees of freedom of the system. It is justified to define a unit time by a Monte Carlo sweep. Note that the more spins the system contains, the higher time resolution we can simulate with.

\subsection{Slip Plane with the Velocity $v$}
Using the introduced time scale, we can also introduce the slip plane with the velocity $v$ to the system with $N$ spins. Corresponding to the setup in Sec.~\ref{sec:SetupModel}, we perform an extended single-spin-flip algorithm as follows:
\begin{enumerate}
	\item \textbf{Shift}: We shift the upper half of the lattice by a lattice constant.
	\item \textbf{Flip}: We perform ordinary single flips for $N/v$ times.
	\item We repeat the processes 1 and 2 for $v$ times.
\end{enumerate}
In the extended algorithm, the upper half slides with the velocity $v$ in a unit time at regular intervals. We proved the fact that this algorithm leads the system of any size to a non-equilibrium stationary state depending on the temperature $T$ and the velocity $v$ (see App.~\ref{chap:ProofEx} for details).

\subsection{Calculation Method}
The observables that we are interested in are the frictional power $P(t)$ and its dissipation rate $D(t)$. In Monte Carlo simulations, they are the energy difference for a unit time due to the shift and the flip operation, respectively. Both observables have the same absolute value in the long-time limit.

