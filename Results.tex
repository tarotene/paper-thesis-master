% !TeX root = Body.tex
\chapter{Results of Simulations}

In this chapter, we show the results of the Monte Carlo simulation of two-dimensional Ising models. To discuss the dependence of the frictional force density $f(L_{z}, T)$ and the bulk energy density $\epsilon_{\rm b}(L_{z}, T)$ on the size $L_{z}$ and the temperature $T$ with the fixed velocity $v=10$, we performed the numerical large size limit $L_{x}\to\infty$ with the fixed $L_{z}$ and $T$ (see appendix \ref{appsec:convcheck}). To get the observables in the non-equilibrium stationary state, we performed the equilibration process for $5000$ sweeps and the stationarization process for $5000$ sweeps for all given parameters. The convergence of the observables to the equilibrium value and the stationary value for these time regions respectively are checked carefully (see appendix \ref{appsec:timeconvcheck}).

The range of parameters in our simulation is as follows. We computed the value of $f(L_{z}, T)$ for temperatures $k_{\rm B}T/J\in\{0.0,0.1,0.2,\dots,1.9,2.0,2.02,2.04,\dots,2.48,2.50,2.6,2.7,\dots,5.0\}$, sizes $L_{z}\in\{4,6,8,10,12,14,16\}$ and these for the two boundary conditions, the  anti-parallel and the parallel. For the anti-parallel boundary the initial state is set to the domain-wall state, where spin variables $\sigma_{i}$ in the upper half of the system are the same value as the upper boundary, and in the lower half as the lower boundary. For the parallel boundary the initial state is set to the magnetized state, where all spin variables $\sigma_{i}$ are the same value as both of boundaries. The reason for these initial states is that these states are the most natural ground state which correpond to each of the boundary conditions.

All the simulations are performed by the single flip algorithm on the Metropolis probability. We performed these simulations for $480$ samples for all parameters and averaged them, and then averaged along the time direction.

In addition, we also show their temperature derivatives to discuss the phase transitions.

\section{Frictional Force Density $f(L_{z}, T)$}

We show the behavior of the frictional force density $f(L_{z}, T)$ (see Figure \ref{fig:fricDens_Allsize}). For both  boundary conditions, the anti-parallel and the parallel, extremely smaller sizes such as $L_{z}=4,6$ make greater differences from its value for $L_{z}=16$, where an asymptotic behavior emerges. We can expect that for more larger sizes such as $L_{z}=18,20,\dots$ makes no longer great differences from that of $L_{z}=16$, thus we can say that the system reaches \textit{two-dimension} in the vicinity of the size $L_{z}=16$.

An additional important aspect is the difference between the values for anti-parallel and the parallel boundary conditions. In the former case, the smaller size $L_{z}$ makes the greater value of the frictional force density $f(L_{z}, T)$. On the other hand, the latter case indicates the reverse behavior that smaller size $L_{z}$ makes the smaller value of the frictional force density $f(L_{z}, T)$. This physical meaning is that the size $L_{z}$ and the correlation length of the system along the $z$-direction $\xi_{z}(\beta)$ become comparable, and then the system behaves as one-dimensional one for the smaller size $L_{z}$, whereas much greater size $L_{z}$ than the correlation length $\xi_{z}(\beta)$ makes the system two-dimensional. Note that for $L_{z}=16$ behaviors for both cases are similar to each other and we can expect that these behaviors reach to the results by Kadau\cite{Kadau2008}.

\begin{figure}[htbp]
	\centering
	\subcaptionbox{Anti-parallel boundary condition\label{fig:fricDens_Allsize_AP}}{\includegraphics[width=0.45\linewidth]{../../NumCalc/ClassicalSpinMC/fricDensP_Allsize_AP.eps}}
	\subcaptionbox{Parallel boundary condition\label{fig:fricDens_Allsize_P}}{\includegraphics[width=0.45\linewidth]{../../NumCalc/ClassicalSpinMC/fricDensP_Allsize_P.eps}}
	
	\caption{The temperature $T$ dependence of the frictional force density $f(L_{z}, T)$ with the parallel boundary condition along the $z$-direction for each longitudinal size $L_{z}$ are plotted.}
	\label{fig:fricDens_Allsize}
\end{figure}

We additionally show the behavior of Its temperature derivative $\partial f(L_{z}, T)/\partial T$ (see Figure \ref{fig:dFricDens_Allsize}). These exhibits the crossover-driven divergence at characteristic temperatures, near $T=2.25$ for the anti-parallel boundary and near $T=2.50$ for the parallel boundary to reflect the steeper slopes of the frictional force density $f(L_{z}, T)$. If we regards the peak $T_{\rm peak}(L_{z})$ as the pseudo critical point for all sizes $L_{z}=4,6,8,10,12,14,16$, where $\partial^{2} f(L_{z}, T)/\partial T^{2}|_{T=T_{\rm peak}(L_{z})} = 0$, we can recognize the shift of peaks toward the higher temperature for the anti-parallel boundary, whereas peaks shifts toward the lower for the parallel boundary. These describes the effects of the boundary condition which acts on the system as an effective field, and the effects are enhanced by the smaller sizes $L_{z}$. Namely the anti-parallel boundary acts as a demagnetizing field such that the pseudo critical point $T_{\rm peak}(L_{z})$ shifts toward the lower, and the parallel boundary acts as a magnetizing field such that the pseudo critical point $T_{\rm peak}(L_{z})$ shifts toward the higher.

Remarkably peaks $T_{\rm peak}(L_{z})$ for both boundaries can be expected to hit the temperature higher than the ordinary critical point $T_{\rm c}=\text{\textcolor{red}{exact value or equation form}}$. This implies that both the ordinary phase transition and the non-equilibrium phase transition occur at two different temperatures.

\begin{figure}[htbp]
	\centering
	\subcaptionbox{Anti-parallel boundary condition\label{fig:dFricDens_Allsize_AP}}{\includegraphics[width=0.45\linewidth]{../../NumCalc/ClassicalSpinMC/dFricDensP_Allsize_AP.eps}}
	\subcaptionbox{Parallel boundary condition\label{fig:dFricDens_Allsize_P}}{\includegraphics[width=0.45\linewidth]{../../NumCalc/ClassicalSpinMC/dFricDensP_Allsize_P.eps}}
	
	\caption{The temperature $T$ dependence of the temperature derivative $\partial f(L_{z}, T)/\partial T$ with the parallel boundary condition along the $z$-direction for each longitudinal size $L_{z}$ are plotted.}
	\label{fig:dFricDens_Allsize}
\end{figure}

\section{Bulk Energy Density $\epsilon_{\rm b}(L_{z}, T)$}

We show the behavior of the bulk energy density $\epsilon_{\rm b}(L_{z}, T)$ (see Figure \ref{fig:EnDens_Allsize}). As same as frictional force densities $f(L_{z}, T)$, bulk energy densities $\epsilon_{\rm b}(L_{z}, T)$ indicates an asymtotic behavior.

In the anti-parallel case, the smaller size $L_{z}$ makes the system well disorder. On the other hand, the parallel case indicates no drastic change driven by the smaller size $L_{z}$.

\begin{figure}[htbp]
	\centering
	\subcaptionbox{Anti-parallel boundary condition\label{fig:EnDens_Allsize_AP}}{\includegraphics[width=0.45\linewidth]{../../NumCalc/ClassicalSpinMC/EnDens_Allsize_AP.eps}}
	\subcaptionbox{Parallel boundary condition\label{fig:EnDens_Allsize_P}}{\includegraphics[width=0.45\linewidth]{../../NumCalc/ClassicalSpinMC/EnDens_Allsize_P.eps}}
	
	\caption{The temperature $T$ dependence of the bulk energy density $\epsilon_{\rm b}(L_{z}, T)$ with the parallel boundary condition along the $z$-direction for each longitudinal size $L_{z}$ are plotted.}
	\label{fig:EnDens_Allsize}
\end{figure}

We additionally show the behavior of Its temperature derivative $c_{\rm b}(L_{z},T)=\partial \epsilon_{\rm b}(L_{z}, T)/\partial T$ (see Figure \ref{fig:dFricDens_Allsize}). The behavior is similar to that of the temperature derivative of the frictional force density $\partial f(L_{z}, T)/\partial T$ in terms of the peak $T_{\rm peak}(L_{z})$ for enough larger sizes.

\begin{figure}[htbp]
	\centering
	\subcaptionbox{Anti-parallel boundary condition\label{fig:dEnDens_Allsize_AP}}{\includegraphics[width=0.45\linewidth]{../../NumCalc/ClassicalSpinMC/dEnDens_Allsize_AP.eps}}
	\subcaptionbox{Parallel boundary condition\label{fig:dEnDens_Allsize_P}}{\includegraphics[width=0.45\linewidth]{../../NumCalc/ClassicalSpinMC/dEnDens_Allsize_P.eps}}
	
	\caption{The temperature $T$ dependence of the bulk energy density $\epsilon(L_{z}, T)$ with the parallel boundary condition along the $z$-direction for each longitudinal size $L_{z}$ are plotted.}
	\label{fig:dEnDens_Allsize}
\end{figure}